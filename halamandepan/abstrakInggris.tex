%-----------------------------------------------------------------------------------------------%
%
% Maret 2019
% Template Latex untuk Tugas Akhir Program Studi Sistem informasi ini
% dikembangkan oleh Inggih Permana (inggihjava@gmail.com)
%
% Template ini dikembangkan dari template yang dibuat oleh Andreas Febrian (Fasilkom UI 2003).
%
% Orang yang cerdas adalah orang yang paling banyak mengingat kematian.
%
%-----------------------------------------------------------------------------------------------%

\chapter*{\MakeUppercase{\textit{\judulInggris}}}

\fontsize{14}{16.8}
\begin{center}
	\vspace{0.3cm}
	\MakeUppercase{\textbf{\penulis}}\\
	\MakeUppercase{\textbf{NIM: \nim}}\\
	\fontsize{12}{14.4}
	\vspace{0.7cm}

	\textit{Date of Final Exam: \tanggalSidangInggris}\\
	\textit{Graduation Period:}\ \ \ \ \ \ \ \ \ \ \ \ \ \ \ \ \ \ \ \ \ \ \ \ \ \ \ \ \ \ \ \ \ \ \ \ \ \

	\vspace{0.7cm}
	\emph{Department of \programStudiInggris}\\
	\textit{\fakultasInggris}\\
	\universitasInggris\\
	\alamatUniversitasInggris\\

	\vspace{0.7cm}
\end{center}

\fontsize{12}{14.4}
\begin{center}\MakeUppercase{\textbf{\emph{Abstract}}}\end{center}

\noindent
\fontsize{10pt}{12pt}\selectfont
\emph{Laboratories serve an important role in educational institutions by supporting practical activities and research projects. Efficient management is required to guarantee the best use of resources and facilities. This study aims to develop the Laboratory Inventory Information System (SITARIS SI) into an Integrated Laboratory Management System for the Information Systems Study Program at UIN Sultan Syarif Kasim Riau. By employing the Agile Development methodology, the study addresses various limitations of the previous SITARIS SI system, such as coding errors, malfunctioning loan features, and report format inconsistencies. The new system is enhanced with additional features, including laboratory scheduling management, to effectively meet user needs. Integration with other supporting systems, such as the Laboratory Visitor Information System (LABVIS) and the Laboratory Assistant Registration Information System (LARIS), further strengthens the system's capabilities. The findings indicate that the Integrated Laboratory Management System improves operational efficiency, simplifies inventory management, and enhances overall laboratory governance. Moreover, the system provides accurate and tailored reports for laboratory heads, thereby supporting more informed decision-making processes. This study thus makes a significant contribution to optimizing laboratory management within the Information Systems Study Program environment.}\\
\noindent{\emph{\textbf{Keywords:} Agile Development, Integrated Laboratory Management System, Integration, Inventory Management, Laboratory Governance}} \\