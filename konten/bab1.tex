%-----------------------------------------------------------------------------------------------%
%
% Maret 2019
% Template Latex untuk Tugas Akhir Program Studi Sistem informasi ini
% dikembangkan oleh Inggih Permana (inggihjava@gmail.com)
%
% Template ini dikembangkan dari template yang dibuat oleh Andreas Febrian (Fasilkom UI 2003).
%
% Orang yang cerdas adalah orang yang paling banyak mengingat kematian.
%
%-----------------------------------------------------------------------------------------------%

%-----------------------------------------------------------------------------%
\chapter{\babSatu}
%-----------------------------------------------------------------------------%

%-----------------------------------------------------------------------------%
\section{Latar Belakang}
%-----------------------------------------------------------------------------%
Laboratorium merupakan salah satu fasilitas vital dalam institusi pendidikan yang berperan penting dalam mendukung kegiatan praktikum dan penelitian \cite{la2021comparison}. Tata kelola laboratorium yang baik menjadi kunci dalam memastikan penggunaan peralatan dan fasilitas secara optimal dan efisien \cite{jeffrey_s__warren_2017}. Pengelolaan inventaris yang akurat dan efisien adalah bagian integral dari tata kelola ini, yang bertujuan untuk membangun budaya kualitas dalam pendidikan tinggi \cite{abrantes2020governance}. Dengan tata kelola yang efektif, institusi dapat memastikan bahwa semua sumber daya laboratorium digunakan secara maksimal untuk mendukung pengembangan ilmu pengetahuan dan teknologi.

Universitas Islam Negeri Sultan Syarif Kasim Riau sebagai salah satu perguruan tinggi negeri di Indonesia memiliki Program Studi Sistem Informasi di bawah naungan Fakultas Sains dan Teknologi \cite{uin-suska-website}. Program Studi ini dilengkapi dengan fasilitas laboratorium yang menyeluruh untuk mendukung pelaksanaan Tri Dharma Perguruan Tinggi yaitu, pengajaran, penelitian, dan pengabdian kepada masyarakat. Ketiga pilar ini berperan sinergis dalam menjadikan UIN Suska Riau sebagai kontributor signifikan dalam pengembangan ilmu pengetahuan, teknologi, dan kemajuan masyarakat.

Sejak tahun 2002, Program Studi Sistem Informasi telah mengelola tiga laboratorium terpadu di bawah Fakultas Sains dan Teknologi, yaitu Laboratorium Rekayasa Sistem Informasi (RSI), Laboratorium Internet (INT), dan Laboratorium Software Engineering (SE) \cite{lab-si-website}. Laboratorium-laboratorium ini berfungsi tidak hanya sebagai sarana praktikum bagi mahasiswa sesuai kurikulum, tetapi juga sebagai pusat kegiatan riset dan inovasi yang memberikan manfaat substansial bagi civitas akademika, termasuk mahasiswa dan dosen.

Laboratorium-laboratorium Program Studi Sistem Informasi dilengkapi dengan fasilitas yang memadai untuk mendukung pembelajaran mahasiswa dan berbagai kegiatan akademik lainnya. Untuk memastikan efektivitas dan efisiensi tata kelola laboratorium, dilakukan evaluasi berkala terhadap seluruh aspek fasilitas yang ada \cite{lab-si-website}. Tata kelola laboratorium yang baik sangat penting untuk memantau dan mengelola penggunaan peralatan serta fasilitas laboratorium secara optimal sehingga dapat mendukung kegiatan akademik dan penelitian dengan lebih baik \cite{dongapure_a__c___2024}.

Dalam konteks Program Studi Sistem Informasi UIN Suska Riau, tata kelola laboratorium sudah dilakukan dengan beberapa cara yaitu mulai dari pengelolaan inventaris laboratorium sebelumnya dilakukan secara manual, yang mengakibatkan berbagai kendala seperti kesulitan dalam pemantauan dan pengelolaan data inventaris, serta ketidakefisienan dalam pengolahan data. Studi oleh Wild \citeyear{smith2021agile} menunjukkan bahwa pengelolaan inventaris yang tidak efisien dapat menghambat operasional laboratorium \cite{wild2017best}. Dalam mengatasi permasalahan tersebut, laboratorium program studi sistem informasi telah mengimplementasikan sistem informasi inventaris bernama SITARIS SI, yang dikembangkan menggunakan framework CodeIgniter 4 sebagai bagian dari penelitian kerja praktek mini proyek tahun 2023. Tata kelola laboratorium dalam hal kunjungan juga sudah diterapkan sistem informasi kunjungan bernama Laboratory Visitor Information System yang disingkat (LABVIS) pada tahun 2023 yang bertujuan untuk mempermudah pemantauan dan pengelolaan kunjungan laboratorium secara efisien. Serta sedang dalam proses pengembangan sebuah sistem informasi pendaftaran asisten laboratorium yang bernama Laboratory Assistans Registration Information System yang disingkat (LARIS). Hal ini mendukung tujuan laboratorium Program Studi Sistem Informasi dalam mewujudkan sistem informasi manajemen laboratorium terintegrasi yang bernama Integrated Laboratory Management Information System yang disingkat sebagai (ILMIS) \cite{lab-si-website}.

Meskipun SITARIS SI telah berperan penting dalam tata kelola laboratorium, perkembangan teknologi yang pesat dan kebutuhan yang semakin kompleks telah memunculkan berbagai tantangan baru. Beberapa permasalahan yang teridentifikasi pada SITARIS SI meliputi kesalahan dalam pembuatan kode barang, disfungsi fitur peminjaman barang dan juga ruangan, ketidaksesuaian laporan akhir pada SITARIS SI dengan format laporan akhir kepala laboratorium, serta berbagai kekurangan lainnya, menjadikan sistem tersebut tidak mampu secara optimal dalam mendukung kebutuhan tata kelola laboratorium.

Dalam upaya mengatasi tantangan tersebut, pengembangan sistem diperlukan untuk meningkatkan kualitas dan fungsionalitas SITARIS SI. Pengembangan sistem akan berfokus pada peningkatan efisiensi pengelolaan laboratorium dengan memperhatikan umpan balik dari pengguna secara berkelanjutan.

Berdasarkan latar belakang tersebut, penelitian ini bertujuan untuk mengembangkan Sistem Informasi Inventaris Laboratorium (SITARIS SI) Program Studi Sistem Informasi UIN Suska Riau menjadi Sistem Manajemen Laboratorium yang lebih menyeluruh. Pengembangan sistem akan berfokus pada peningkatan fungsionalitas dan kinerja sistem secara keseluruhan, sehingga mampu memenuhi kebutuhan tata kelola laboratorium.
%-----------------------------------------------------------------------------%
\section{Rumusan Masalah}
%-----------------------------------------------------------------------------%
Berdasarkan latar belakang yang telah diuraikan, diperoleh rumusan masalah untuk penelitian ini adalah bagaimana pengembangan lebih lanjut SITARIS SI menjadi Sistem Manajemen Laboratorium yang lebih menyeluruh menggunakan metode Agile Development.

%-----------------------------------------------------------------------------%
\section{Batasan Masalah}
%-----------------------------------------------------------------------------%
Dalam melakukan penelitian diperlukan batasan agar tidak menyimpang dari apa yang direncanakan. Adapun batasan masalah dalam penelitian Tugas Akhir ini adalah sebagai berikut:
\begin{enumerate}
	\item Penelitian ini hanya akan fokus pada pengembangan Sistem Informasi Inventaris Laboratorium (SITARIS SI) menjadi Sistem Manajemen Laboratorium.
	\item Pengembangan sistem akan menggunakan metode Agile Development.
	\item Penelitian ini tidak mencakup pengembangan perangkat keras laboratorium.
	\item Evaluasi sistem akan dilakukan berdasarkan umpan balik dari pengguna di Program Studi Sistem Informasi UIN Suska Riau.
	\item Penelitian ini hanya akan mencakup laboratorium yang berada di bawah naungan Program Studi Sistem Informasi UIN Suska Riau.
\end{enumerate}

%-----------------------------------------------------------------------------%
\section{Tujuan}
%-----------------------------------------------------------------------------%
Tujuan Tugas Akhir ini adalah:
\begin{enumerate}
	\item Menganalisis SITARIS SI sebagai Sistem Informasi Inventaris Laboratorium untuk mengetahui kekurangan dan kendala yang ada.
	\item Mengembangkan Sistem Informasi Inventaris Laboratorium (SITARIS SI) menjadi Sistem Manajemen Laboratorium.
\end{enumerate}

%-----------------------------------------------------------------------------%
\section{Manfaat}
%-----------------------------------------------------------------------------%
Hasil penelitian diharapkan dapat menjadi sebuah Sistem Informasi Manajemen Laboratorium yang terintegrasi yang memberikan kemudahan dalam tata kelola laboratorium. Sistem ini diharapkan mampu mengelola tata kelola laboratorium dengan lebih efisien, memantau penggunaan peralatan dan fasilitas secara real-time, serta menyediakan laporan yang akurat dan sesuai dengan kebutuhan kepala laboratorium. Selain itu, sistem ini juga diharapkan dapat memfasilitasi peminjaman barang dan ruangan dengan lebih mudah, serta mengelola kunjungan laboratorium secara efektif. Dengan demikian, sistem ini akan mendukung kegiatan praktikum dan penelitian dengan lebih baik, serta meningkatkan kualitas pengelolaan laboratorium secara keseluruhan.
%-----------------------------------------------------------------------------%
\section{Sistematika Penulisan}
%-----------------------------------------------------------------------------%
Sistematika penulisan laporan adalah sebagai berikut:

\textbf{BAB 1. \babSatu}

BAB 1 pada tugas akhir ini berisi tentang: (1) Latar Belakang masalah; (2) Rumusan Masalah; (3) Batasan Masalah; (4) Tujuan; (5) Manfaat; dan (6) Sistematika Penulisan.

\textbf{BAB 2. \babDua}

BAB 2 pada Tugas Akhir ini berisi tentang: (1) Profil Instansi; (2) Laboratorium; (3) SITARIS SI; (4) Model Pengambangan Sistem; (5) Observasi; (6) Web; (7) Framework; (8) CodeIgniter; (9) Database; (10) MariaDB; (11) PHP; (12) XAMPP.

\textbf{BAB 3. \babTiga}

BAB 3 pada Tugas Akhir ini berisi tentang: (1) Tahap Perencanaan; (2) Tahap Pengumpulan Data; (3) Tahap Analisis dan Perancangan; (4) Tahap Implementasi dan Pengujian; (5) Tahap Dokumentasi.

\textbf{BAB 4. \babEmpat}

BAB 4 pada Tugas Akhir ini berisi tentang: (1) Analisis SITARIS SI; (2) Perancangan Sistem; (3) Implementasi Sistem; (4) Pengujian Sistem; (5) Dokumentasi Sistem.

\textbf{BAB 5. \babLima}

BAB 5 pada Tugas Akhir ini berisi tentang: (1) Kesimpulan; (2) Saran.
