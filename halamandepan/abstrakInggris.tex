%-----------------------------------------------------------------------------------------------%
%
% Maret 2019
% Template Latex untuk Tugas Akhir Program Studi Sistem informasi ini
% dikembangkan oleh Inggih Permana (inggihjava@gmail.com)
%
% Template ini dikembangkan dari template yang dibuat oleh Andreas Febrian (Fasilkom UI 2003).
%
% Orang yang cerdas adalah orang yang paling banyak mengingat kematian.
%
%-----------------------------------------------------------------------------------------------%

\chapter*{\MakeUppercase{\textit{\judulInggris}}}

\fontsize{14}{16.8}
\begin{center}
	\vspace{0.3cm}
	\MakeUppercase{\textbf{\penulis}}\\
	\MakeUppercase{\textbf{NIM: \nim}}\\
	\fontsize{12}{14.4}
	\vspace{0.7cm}

	\textit{Date of Final Exam: \tanggalSidangInggris}\\
	\textit{Graduation Period:}\ \ \ \ \ \ \ \ \ \ \ \ \ \ \ \ \ \ \ \ \ \ \ \ \ \ \ \ \ \ \ \ \ \ \ \ \ \

	\vspace{0.7cm}
	\emph{Department of \programStudiInggris}\\
	\textit{\fakultasInggris}\\
	\universitasInggris\\
	\alamatUniversitasInggris\\

	\vspace{0.7cm}
\end{center}

\fontsize{12}{14.4}
\begin{center}\MakeUppercase{\textbf{\emph{Abstract}}}\end{center}

\noindent
\fontsize{10pt}{12pt}\selectfont
\emph{Laboratories play a crucial role in educational institutions by supporting practical activities and research. Efficient management is essential to ensure the optimal use of resources and facilities. This study aims to develop the Laboratory Inventory Information System (SITARIS) into an Integrated Laboratory Management System for the Information Systems Study Program at UIN Sultan Syarif Kasim Riau. The study emphasizes the use of Agile Development as the primary approach in system development. Various limitations in the previous SITARIS SI system were identified and resolved, including coding errors, malfunctioning loan features, and mismatched report formats. The resulting system is not only more stable but also enhanced with additional features such as laboratory schedule management. Integration with other supporting systems, such as the Laboratory Visitor Information System (LABVIS) and the Laboratory Assistant Registration Information System (LARIS), which previously operated separately, was also achieved. The results show that the Integrated Laboratory Management System can improve operational efficiency, streamline inventory management, and enhance overall laboratory governance. Additionally, the system provides accurate reports tailored to the needs of the head of the laboratory, thus supporting better decision-making processes. Therefore, this research makes a significant contribution to optimizing laboratory governance within the Information Systems Study Program.}\\
\noindent{\emph{\textbf{Keywords:} Agile Development, Integrated Laboratory Management System, Integration, Inventory Management, Laboratory Governance}} \\