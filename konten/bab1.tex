%-----------------------------------------------------------------------------------------------%
%
% Maret 2019
% Template Latex untuk Tugas Akhir Program Studi Sistem informasi ini
% dikembangkan oleh Inggih Permana (inggihjava@gmail.com)
%
% Template ini dikembangkan dari template yang dibuat oleh Andreas Febrian (Fasilkom UI 2003).
%
% Orang yang cerdas adalah orang yang paling banyak mengingat kematian.
%
%-----------------------------------------------------------------------------------------------%

%-----------------------------------------------------------------------------%
\chapter{\babSatu}
%-----------------------------------------------------------------------------%

%-----------------------------------------------------------------------------%
\section{Latar Belakang}
%-----------------------------------------------------------------------------%
Program Studi Sistem Informasi merupakan salah satu program studi yang berada di Fakultas Sains dan Teknologi Universitas Islam Negeri Sultan Syarif Kasim Riau. Program Studi Sistem Informasi dilengkapi dengan laboratorium yang mendukung pelaksanaan Tri Dharma Perguruan Tinggi yang dalam konteks pendidikan tinggi di Indonesia, meliputi pengajaran, penelitian, dan pengabdian kepada masyarakat. Pilar-pilar ini secara kolektif menjadikan perguruan tinggi sebagai kontributor dalam pengembangan pengetahuan, teknologi, dan masyarakat secara keseluruhan. Hal ini juga mencakup sesi praktikum yang bermanfaat bagi mahasiswa dan dosen.

Laboratorium merupakan tempat yang digunakan mahasiswa untuk melakukan kegiatan pengujian, riset ilmiah, praktikum, serta penelitian. Program Studi Sistem Informasi memiliki fasilitas infrastruktur pendukung Tridharma Perguruan Tinggi yang baik, salah satunya adalah laboratorium terpadu di bawah Fakultas Sains dan Teknologi yang dikelola oleh Program Studi Sistem Informasi sejak tahun 2002. Terdapat tiga laboratorium yang dikelola oleh Program Studi Sistem Informasi, yaitu Laboratorium Rekayasa Sistem Informasi (RSI), Laboratorium Internet (INT), dan Laboratorium Software Engineering (SE) \cite{lab-si-website}. Ketiga laboratorium tersebut merupakan sumber daya berharga yang dapat dimanfaatkan secara efektif untuk mencapai tujuan universitas dan menghasilkan lulusan yang kompeten di Program Studi Sistem Informasi melalui pendidikan, penelitian, dan pengabdian kepada masyarakat, dengan tetap mengintegrasikan nilai-nilai keislaman. Laboratorium-laboratorium ini tidak hanya digunakan untuk praktikum mahasiswa sesuai kurikulum, tetapi juga mendukung berbagai kegiatan mahasiswa dan dosen untuk meningkatkan pengetahuan di bidang Sistem Informasi. Laboratorium-laboratorium Program Studi Sistem Informasi dilengkapi dengan fasilitas yang memadai untuk mendukung pembelajaran mahasiswa. Evaluasi terhadap fasilitas yang ada di laboratorium-laboratorium Program Studi Sistem Informasi dilakukan untuk meningkatkan pengalaman belajar mahasiswa dalam memahami materi, termasuk manajemen inventaris. Manajemen inventaris merupakan salah satu bentuk pengawasan barang - barang yang ada di Laboratorium Program Studi Sistem Informasi di UIN Suska Riau. Tujuan dari manajemen ini untuk memantau jumlah, kondisi, dan status barang yang ada di laboratorium. 

Berdasarkan penelitian sebelumnya, proses pengelolaan inventaris yang mulanya masih dilakukan secara manual, dengan pencatatan yang belum terkomputerisasi yang seringkali mengakibatkan kesulitan dalam memantau dan mengelola data inventaris serta membuat pengolahan data menjadi tidak mudah dan tidak efisien. Masalah ini sudah diatasi dengan solusi mengimplementasikan sistem informasi inventaris yang disebut SITARIS SI di Laboratorium Program Studi Sistem Informasi menggunakan \textit{framework} CodeIgniter 4 yang merupakan bagian dari penelitian kerja praktek mini proyek sebelumnya. Sistem SITARIS SI telah memberikan kontribusi signifikan dalam pengelolaan inventaris laboratorium Program Studi Sistem Informasi. Namun, seiring dengan perkembangan teknologi dan kebutuhan pengguna yang semakin kompleks dan dalam penerapannya ditemukan beberapa masalah seperti kesalahan pembuatan kode barang, tidak berfungsinya beberapa fitur seperti peminjaman barang dan ruangan serta kebutuhan pada beberapa aspek, maka diperlukan evaluasi menyeluruh terhadap kualitas sistem tersebut untuk memastikan efektivitas dan efisiensinya dalam jangka panjang. Salah satu model evaluasi yang dapat digunakan untuk menilai kualitas perangkat lunak adalah ISO 9126. Model ini menyediakan kerangka kerja yang komprehensif untuk mengukur kualitas sistem informasi berdasarkan enam karakteristik utama, yaitu: \textit{efficiency, functionality, maintainability portability, reliability dan usability} \cite{iso2001iec}.

Berbagai penelitian terdahulu telah membuktikan keefektifan dan fleksibilitas model ISO 9126 dalam evaluasi sistem informasi di berbagai konteks. Dwiyantoro \citeyear{dwiyantoro2020evaluasi} dan Melathi \citeyear{melathi2017penerapan} mendemonstrasikan bagaimana model ini dapat diterapkan untuk mengevaluasi sistem informasi akademik dan perpustakaan, menghasilkan pengetahuan yang berharga untuk perbaikan dan pengembangan sistem. Rohman \citeyear{rohman2022evaluasi} lebih lanjut menunjukkan adaptabilitas ISO 9126 dalam fokus pada aspek-aspek spesifik seperti \textit{usability} dan \textit{functionality}. Keberhasilan penerapan ISO 9126 dalam berbagai konteks ini menegaskan kesesuaiannya untuk evaluasi Sistem Informasi Inventaris Laboratorium (SITARIS SI) Program Studi Sistem Informasi UIN Suska Riau, yang memerlukan penilaian komprehensif terhadap berbagai aspek kualitas perangkat lunak.

Selain itu, penerapan model ISO 9126 untuk evaluasi SITARIS SI sejalan dengan visi Program Studi Sistem Informasi UIN Suska Riau dalam mengintegrasikan teknologi informasi terkini dengan nilai-nilai keislaman. Evaluasi ini dapat menjadi langkah konkret dalam upaya peningkatan mutu layanan pendidikan dan pengelolaan sumber daya universitas yang transparan dan akuntabel.

Berdasarkan latar belakang tersebut, penelitian ini bertujuan untuk menerapkan model ISO 9126 dalam mengevaluasi kualitas Sistem Informasi Inventaris Laboratorium (SITARIS SI) Program Studi Sistem Informasi UIN Suska Riau. Hasil evaluasi ini diharapkan dapat memberikan rekomendasi yang berharga untuk perbaikan dan pengembangan sistem di masa mendatang, serta menjadi acuan bagi pengembangan sistem informasi serupa di lingkungan akademik lainnya.

%-----------------------------------------------------------------------------%
\section{Perumusan Masalah}
%-----------------------------------------------------------------------------%
Berdasarkan latar belakang yang telah diuraikan, diperoleh rumusan masalah untuk penelitian ini sebagai berikut:
\begin{enumerate}
	\item Bagaimana mengevaluasi kualitas SITARIS SI menggunakan metode ISO 9126 untuk dapat menentukan tingkat kualitas sistem dan kemudahan akses bagi pengguna serta memberikan rekomendasi untuk pengembangan sistem selanjutnya.
	\item Apakah SITARIS SI sudah memenuhi aspek pada standar kualitas mutu perangkat lunak yang baik berdasarkan metode ISO 9126.
\end{enumerate}

%-----------------------------------------------------------------------------%
\section{Batasan Masalah}
%-----------------------------------------------------------------------------%
Dalam melakukan penelitian diperlukan batasan agar tidak menyimpang dari apa yang direncanakan. Adapun batasan masalah dalam penelitian Tugas Akhir, yaitu:
\begin{enumerate}
	\item Penelitian ini menggunakan Metode ISO 9126 untuk mengukur tingkat kualitas SITARIS SI.
	\item Evaluasi pada SITARIS SI menggunakan 6 faktor, yaitu: \textit{Functionality, Reliability, Usability, Efficiency, Portability,} dan \textit{Maintainability.} 
	\item Teknik pengumpulan data menggunakan Kuesioner dan Wawancara.
	\item Teknik pengujian kualitas menggunakan Black Box Testing, dan Manual Testing.
	\item Teknik sampling yang digunakan adalah \textit{Probability Sampling} yaitu dengan \textit{Proportionate Stratified Random sampling}.
\end{enumerate}

%-----------------------------------------------------------------------------%
\section{Tujuan}
%-----------------------------------------------------------------------------%
Tujuan Tugas Akhir ini adalah:
\begin{enumerate}
	\item Untuk mengetahui tingkat kualitas website SITARIS SI menggunakan Metode ISO 9126 pada faktor \textit{Functionality, Reliability, Usability, Efficiency, Portability,} dan \textit{Maintainability.} 
	\item Membuat rekomendasi solusi berdasarkan hasil evaluasi kualitas website SITARIS SI guna meningkatkan kualitas website pengelolaan inventaris di Laboratorium Sistem Informasi.	
\end{enumerate}

%-----------------------------------------------------------------------------%
\section{Manfaat}
%-----------------------------------------------------------------------------%
Hasil penelitian diharapkan dapat menjadi bahan pertimbangan bagi pihak Laboratorium Sistem Informasi guna melakukan perbaikan terhadap kualitas website apabila selama diterapkan website ini masih terdapat kekurangan.

%-----------------------------------------------------------------------------%
\section{Sistematika Penulisan}
%-----------------------------------------------------------------------------%
Sistematika penulisan laporan adalah sebagai berikut:

\textbf{BAB 1. \babSatu}

BAB 1 pada tugas akhir ini berisi tentang: (1) Latar Belakang masalah; (2) Rumusan Masalah; (3) Batasan Masalah; (4) Tujuan; (5) Manfaat; dan (6) Sistematika Penulisan.

\textbf{BAB 2. \babDua}

BAB 2 pada Tugas Akhir ini berisi tentang: (1) Penelitian Terdahulu; (2) Profil Instansi; (3) Sistem Informasi Inventaris; (4) Laboratorium; (5) SITARIS SI; (6)Evaluasi; (7) Kualitas Sistem; (8) Model ISO 9126; (9) Skala Likert; (10) Observasi; (11) Kuisioner; (12) Populasi dan Sampel; (13) SPSS.


\textbf{BAB 3. \babTiga}

BAB 3 pada Tugas Akhir ini berisi tentang: (1) Tahap Perencanaan; (2)
Tahap Pengumpulan Data; (3) Tahap Analisis dan Hasil; (4) Tahap Kesimpulan dan Dokumentasi.


\textbf{BAB 4. \babEmpat}

BAB 4 pada Tugas Akhir ini berisi tentang: (1) Analisis Kondisi SITARIS SI Saat ini; (2) Analisis Pengolahan Data Kuesioner; (3) Analisis Uji Validitas dan Reliabilitas; (4) Analisis Pengukuran Model ISO 9126; (5) Analisis Kualitas SITARIS SI; (6) Analisis Hasil Penelitian; (7) Rekomendasi.

\textbf{BAB 5. \babLima}

BAB 5 pada Tugas Akhir ini berisi tentang: (1) Kesimpulan; (2) Saran.