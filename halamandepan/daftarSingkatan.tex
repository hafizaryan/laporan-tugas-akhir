%-----------------------------------------------------------------------------------------------%
%
% Maret 2019
% Template Latex untuk Tugas Akhir Program Studi Sistem informasi ini
% dikembangkan oleh Inggih Permana (inggihjava@gmail.com)
%
% Template ini dikembangkan dari template yang dibuat oleh Andreas Febrian (Fasilkom UI 2003).
%
% Orang yang cerdas adalah orang yang paling banyak mengingat kematian.
%
%-----------------------------------------------------------------------------------------------%

%-----------------------------------------------------------------------------%
\chapter*{DAFTAR SINGKATAN}
%-----------------------------------------------------------------------------%
\begin{tabular}{lll}
	DB        & : & \textit{Database}                                             \\
	FST       & : & Fakultas Sains dan Teknologi                                  \\
	Kaprodi   & : & Kepala Program Studi                                          \\
	LABVIS    & : & \textit{Laboratory Visitor Information System}                \\
	LARIS     & : & \textit{Laboratory Assistant Registration Information System} \\
	PHP       & : & \textit{Hypertext Preprocessor}                               \\
	PRODI     & : & Program Studi                                                 \\
	QR        & : & \textit{Quick Response}                                       \\
	SI        & : & Sistem Informasi                                              \\
	SIKAPE    & : & Sistem Kerja Praktek                                          \\
	SIREPO    & : & Sistem Repositori Laboratorium                                \\
	SITARIS   & : & Sistem Informasi Inventaris Laboratorium                      \\
	SITASI    & : & Sistem Informasi Tugas Akhir                                  \\
	SKS       & : & Satuan Kredit Semester                                        \\
	TA        & : & Tugas Akhir                                                   \\
	UIN SUSKA & : & Universitas Islam Negeri Sultan Syarif Khasim                 \\
	UML       & : & Unified Modelling Language                                    \\
\end{tabular}