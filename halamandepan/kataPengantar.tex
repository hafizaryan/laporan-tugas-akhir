%-----------------------------------------------------------------------------------------------%
%
% Maret 2019
% Template Latex untuk Tugas Akhir Program Studi Sistem informasi ini
% dikembangkan oleh Inggih Permana (inggihjava@gmail.com)
%
% Template ini dikembangkan dari template yang dibuat oleh Andreas Febrian (Fasilkom UI 2003).
%
% Orang yang cerdas adalah orang yang paling banyak mengingat kematian.
%
%-----------------------------------------------------------------------------------------------%

%-----------------------------------------------------------------------------%
\chapter*{\kataPengantar}
%-----------------------------------------------------------------------------%

\textit{Assalamu’alaikum Warahmatullahi Wabarakatuh}.

\textit{Alhamdulillahi Rabbil’Alamin}, bersyukur kehadirat Allah \textit{Subhanahu Wa Ta’ala} atas segala rahmat dan karunia-Nya sehingga peneliti dapat menyelesaikan Tugas Akhir ini. \textit{Shalawat} beserta salam selalu tercurahkan untuk Nabi Muhammad \textit{Shallallahu ’Alaihi Wa Sallam} dengan mengucapkan \textit{Allahumma Sholli’ala Sayyidina Muhammad Wa’ala Ali Sayyidina Muhammad}. Tugas Akhir ini dibuat sebagai salah satu syarat untuk mendapatkan gelar Sarjana Komputer di Program Studi Sistem Informasi Fakultas Sains dan Teknologi Universitas Islam Negeri Sultan Syarif Kasim Riau. Pada penulisan Tugas Akhir ini, banyak pihak yang telah berperan dalam mendukung dan membantu peneliti. Maka dari itu, ungkapan terima kasih peneliti ucapkan kepada:

\begin{enumerate}
	\item Bapak \rektor\ sebagai Rektor \universitas.
	\item Bapak Dr. Hartono, M.Pd sebagai Dekan \fakultas.
	\item Bapak Eki Saputra, S.Kom., M.Kom sebagai Ketua Program Studi Sistem Informasi dan Penguji I peneliti.
	\item Ibu Siti Monalisa, ST., M.Kom sebagai Sekretaris Program Studi Sistem Informasi.
	\item Bapak T. Khairil Ahsyar, S.Kom., M.Kom sebagai Dosen Pembimbing Tugas Akhir ini yang telah banyak memberikan arahan, masukan, nasihat, serta motivasi dalam penyelesaian Tugas Akhir ini.
	\item Bapak Arif Marsal, Lc., MA sebagai Ketua Sidang peneliti yang telah memberikan arahan dan masukan pada penelitian ini.
	\item Bapak Muhammad Jazman, S.Kom., M.Infosys sebagai Penguji II peneliti yang telah memberikan arahan dan masukan pada penelitian ini.
	\item Ibu Mona Fronita, S.Kom., M.Kom sebagai Dosen Pembimbing Akademik yang telah memberikan dukungan, arahan, dan masukan kepada peneliti dari awal hingga akhir perkuliahan.
	\item Seluruh Dosen Program Studi Sistem Informasi Fakultas Sains dan Teknologi Universitas Islam Negeri Sultan Syarif Kasim Riau yang telah memberikan ilmu dan bimbingan selama masa perkuliahan.
	\item Kedua orang tua peneliti yaitu Ayah Almarhum Ahmad Sofyan Siregar dan Ibu Arni Shopiyah Nasution yang selalu memberikan \textit{do'a}, dukungan, dan semangat kepada peneliti. Terima kasih atas segala keringat, jerih payah pengorbanan, dan kerja keras yang telah diberikan kepada peneliti. Semoga Allah \textit{Subhanahu Wa Ta’ala} membalas semua kebaikan Ayah dan Ibu dengan balasan yang lebih baik.
	\item Adik-adik peneliti yaitu Hanif Luthfi Siregar dan Nia Fanesa Siregar yang selalu memberikan \textit{do'a}, dukungan, dan semangat kepada peneliti.
	\item Saudari Rahma Yulia Fani yang telah banyak memberikan motivasi, dukungan, dan semangat kepada peneliti dalam menyelesaikan Tugas Akhir ini.
	\item Keluarga besar \textit{Study Club Information System Networking Club Research} (ISNC Research). Terima kasih sudah memberikan dukungan untuk menyelasaikan Tugas Akhir ini.
	\item Teman-teman Sistem Informasi Angkatan 2021 terkhusus teman seperjuangan di Kelas D yang selalu mendukung, berbagi informasi, dan membantu peneliti dalam menjalankan masa perkuliahan.
	\item Teman-teman di Bangkit \textit{Academy} dan \textit{Coding Camp} yaitu Dani Harmade, Rahmat Afriyanto, Fajri Nurhadi, Hapid Ramdani, Erliandika Syahputra, Ahmeid Aqeil, dan Novrian Pratama yang telah menjadi bagian dari cerita perkuliahan ini.
	      % \item Teman-teman KKN Kampung Tengah yaitu Dewi, Dini, Fida, Hera, Kira, Putri, Uca, Wahyu, dan Zacky. Terima kasih atas pengalaman berharga yang telah diberikan selama masa KKN.
	\item Seluruh pihak yang tidak dapat disebutkan satu-persatu yang telah memberikan kontribusi secara langsung maupun tidak langsung dalam menyelesaikan Tugas Akhir ini.
\end{enumerate}

Semoga segala \textit{do'a} dan dorongan yang telah diberikan selama ini menjadi amal kebajikan dan mendapat balasan setimpal dari Allah \textit{Subhanahu Wa Ta’ala}. Peneliti menyadari bahwa penulisan Tugas Akhir ini masih terdapat kekurangan dan jauh dari kata sempurna. Peneliti berharap untuk kritik dan saran yang membangun yang dapat disampaikan melalui \textit{email} \href{mailto:12150310904@students.uin-suska.ac.id}{12150310904@students.uin-suska.ac.id} atau \href{mailto:hafizaryansiregar@gmail.com}{hafizaryansiregar@gmail.com} untuk Tugas Akhir ini dan semoga Laporan Tugas Akhir ini bermanfaat bagi kita semua. Akhir kata peneliti ucapkan terima kasih.

\textit{Wassalamu’alaikum Warahmatullahi Wabarakaatuh.}

\vspace*{0.1cm}

% \begin{flushright}
% 	\kota, \tanggalPersetujuan\\
% 	Peneliti,\\
% 	\vspace{2cm}
% 	\textbf{\underline{\penulis}\\
% 		NIM. \nim}

% \end{flushright}

\begin{flushright}
	\kota, \tanggalPersetujuan\\
	Peneliti,\\
	\vspace{2cm}
	\textbf{\underline{\penulis}\\
		\vspace{-0.15cm}
		NIM. \nim}

\end{flushright}
