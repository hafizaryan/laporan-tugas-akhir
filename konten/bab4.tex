%-----------------------------------------------------------------------------------------------%
%
% Maret 2019
% Template Latex untuk Tugas Akhir Program Studi Sistem informasi ini
% dikembangkan oleh Inggih Permana (inggihjava@gmail.com)
%
% Template ini dikembangkan dari template yang dibuat oleh Andreas Febrian (Fasilkom UI 2003).
%
% Orang yang cerdas adalah orang yang paling banyak mengingat kematian.
%
%-----------------------------------------------------------------------------------------------%


%-----------------------------------------------------------------------------%
\chapter{\babEmpat}
%-----------------------------------------------------------------------------%
\section{Analisa Sistem Berjalan}
Dalam proses tata kelola yang berlangsung di laboratorium Program Studi Sistem Informasi, hingga saat ini laboratorium telah menerapkan beberapa sistem informasi untuk mengelola berbagai aspek operasionalnya. Sistem-sistem tersebut meliputi:

\begin{enumerate}
	\item LAB SI \textit{Website} adalah sistem informasi yang berfungsi untuk mengelola informasi terkait laboratorium Program Studi Sistem Informasi, mencakup profil, informasi laboratorium, rilis media, pengumuman, dan galeri kegiatan.

	\item \textit{Laboratory Visitor Information System} yang disingkat LABVIS adalah sistem informasi yang digunakan untuk mengelola data kunjungan masuk dan keluar laboratorium, memungkinkan pemantauan dan pencatatan aktivitas pengunjung secara efisien.

	\item \textit{Laboratory Assistant Registration Information System} yang disingkat LARIS adalah sistem informasi yang digunakan untuk mengelola data pendaftar dan proses rekrutmen asisten laboratorium.

	\item Sistem Informasi Inventaris disingkat yang SITARIS adalah sistem informasi inventarisasi yang memfasilitasi pengelolaan dan pemantauan alat serta barang di laboratorium, meningkatkan efisiensi dalam manajemen inventaris.
\end{enumerate}

Implementasi sistem-sistem ini telah secara signifikan meningkatkan efektivitas dan efisiensi tata kelola laboratorium Program Studi Sistem Informasi, memungkinkan pengelolaan yang lebih terstruktur dan terintegrasi dalam berbagai aspek operasional laboratorium. Namun, beberapa sistem tersebut masih memiliki kekurangan dalam menunjang tata kelola laboratorium, terutama dalam hal penjadwalan. Saat ini, tidak ada sistem informasi yang secara khusus mengelola penjadwalan laboratorium Program Studi Sistem Informasi. Pengelolaan penjadwalan masih dilakukan secara manual dengan melakukan validasi dan pengecekan pada jadwal yang diperoleh dari Ketua Program Studi (KaProdi). Hal ini mengakibatkan ketidaksesuaian dan kurangnya informasi mengenai jadwal praktikum di laboratorium. Oleh karena itu, perlu dilakukan penyempurnaan pada sistem informasi yang ada, khususnya SITARIS, agar dapat memenuhi kebutuhan tata kelola laboratorium dalam hal penjadwalan ruangan. Penyempurnaan ini bertujuan untuk mencapai tujuan laboratorium dalam menerapkan \textit{Integrated Laboratory Management Information System} (ILMIS), yang akan mengintegrasikan seluruh aspek manajemen laboratorium, termasuk penjadwalan, ke dalam satu sistem yang efisien.

\section{Analisa Sistem Usulan}
\begin{enumerate}
	\item Analisa Sistem Usulan Penjadwalan Laboratorium
\end{enumerate}
Pengembangan sistem ini akan menyajikan fitur penjadwalan laboratorium yang dapat digunakan oleh Admin, Kaprodi, Sekprodi, dan Aslab. Alur analisa sistem usulan penjadwalan laboratorium dapat dilihat pada Gambar

\section{Analisa Kebutuhan Sistem}
\subsection{Analisa Kebutuhan Fungsional Sistem}
\subsection{Analisa Kebutuhan Non-Fungsional Sistem}
\section{Perancangan}
\subsection{Perancangan Basis Data}
\subsection{Perancangan Struktur Menu}
\subsection{Perancangan Interface}