%-----------------------------------------------------------------------------------------------%
%
% Maret 2019
% Template Latex untuk Tugas Akhir Program Studi Sistem informasi ini
% dikembangkan oleh Inggih Permana (inggihjava@gmail.com)
%
% Template ini dikembangkan dari template yang dibuat oleh Andreas Febrian (Fasilkom UI 2003).
%
% Orang yang cerdas adalah orang yang paling banyak mengingat kematian.
%
%-----------------------------------------------------------------------------------------------%


%-----------------------------------------------------------------------------%
\chapter{\babEmpat}
%-----------------------------------------------------------------------------%
\section{Analisis Sistem Berjalan}
Dalam proses tata kelola yang berlangsung di laboratorium Program Studi Sistem Informasi, hingga saat ini laboratorium telah menerapkan beberapa sistem informasi untuk mengelola berbagai aspek operasionalnya. Sistem-sistem tersebut meliputi:

\begin{enumerate}
	\item LAB SI \textit{Website} adalah sistem informasi yang berfungsi untuk mengelola informasi terkait laboratorium Program Studi Sistem Informasi, mencakup profil, informasi laboratorium, rilis media, pengumuman, dan galeri kegiatan.

	\item \textit{Laboratory Visitor Information System} yang disingkat LABVIS adalah sistem informasi yang digunakan untuk mengelola data kunjungan masuk dan keluar laboratorium, memungkinkan pemantauan dan pencatatan aktivitas pengunjung secara efisien.

	\item \textit{Laboratory Assistant Registration Information System} yang disingkat LARIS adalah sistem informasi yang digunakan untuk mengelola data pendaftar dan proses rekrutmen asisten laboratorium.

	\item Sistem Informasi Inventaris disingkat yang SITARIS adalah sistem informasi inventarisasi yang memfasilitasi pengelolaan dan pemantauan alat serta barang di laboratorium, meningkatkan efisiensi dalam manajemen inventaris.
\end{enumerate}

Implementasi sistem-sistem ini telah secara signifikan meningkatkan efektivitas dan efisiensi tata kelola laboratorium Program Studi Sistem Informasi, memungkinkan pengelolaan yang lebih terstruktur dan terintegrasi dalam berbagai aspek operasional laboratorium. Namun, beberapa sistem tersebut masih memiliki kekurangan dalam menunjang tata kelola laboratorium, terutama dalam hal penjadwalan. Saat ini, tidak ada sistem informasi yang secara khusus mengelola penjadwalan laboratorium Program Studi Sistem Informasi. Pengelolaan penjadwalan masih dilakukan secara manual dengan melakukan validasi dan pengecekan pada jadwal yang diperoleh dari Ketua Program Studi (KaProdi). Hal ini mengakibatkan ketidaksesuaian dan kurangnya informasi mengenai jadwal praktikum di laboratorium. Oleh karena itu, perlu dilakukan penyempurnaan pada sistem informasi yang ada, khususnya SITARIS, agar dapat memenuhi kebutuhan tata kelola laboratorium dalam hal penjadwalan ruangan. Penyempurnaan ini bertujuan untuk mencapai tujuan laboratorium dalam menerapkan \textit{Integrated Laboratory Management Information System} (ILMIS), yang akan mengintegrasikan seluruh aspek manajemen laboratorium, termasuk penjadwalan, ke dalam satu sistem yang efisien.

\section{Analisis Sistem Usulan}
\begin{enumerate}
	\item Analisis Sistem Usulan Penjadwalan Laboratorium
\end{enumerate}
Pengembangan sistem ini akan menyajikan fitur penjadwalan laboratorium yang dapat digunakan oleh Admin, Kaprodi, Sekprodi, dan Aslab. Alur Analisis sistem usulan penjadwalan laboratorium dapat dilihat pada Gambar

\section{Analisis Kebutuhan Sistem}
\subsection{Analisis Kebutuhan Fungsional Sistem}
Sistem ini dirancang untuk memenuhi berbagai kebutuhan fungsional yang esensial dalam pengelolaan penjadwalan laboratorium. Ini mencakup manajemen jadwal yang fleksibel dan mudah diakses, kemampuan pengelolaan jadwal yang intuitif dengan validasi yang ketat, serta sistem pengelolaan informasi yang memungkinkan pemantauan dan manajemen informasi terkait penggunaan laboratorium. Sistem ini juga mendukung proses validasi yang terstruktur  untuk memantau dan memberitahukan status penggunaan laboratorium kepada pengguna. Pengelolaan akses pengguna yang aman dan integrasi yang lancar dengan sistem internal laboratorium lainnya juga menjadi bagian integral dari fungsi sistem ini, memastikan efisiensi dan transparansi dalam seluruh proses penjadwalan laboratorium.

\subsection{Analisis Kebutuhan Non-Fungsional Sistem}
Kebutuhan non-fungsional sistem terbagi dalam dua kategori utama yaitu kebutuhan perangkat lunak dan kebutuhan perangkat keras. Analisis terhadap kebutuhan perangkat keras dilakukan untuk mengoptimalkan dan mempermudah proses perancangan serta implementasi sistem yang akan dibangun.
% -----------------------------------------------------------------------------%
\begin{enumerate}
	\item  Analisis Kebutuhan Perangkat Lunak \\
	      Pada tahap analisis ini, peneliti mengidentifikasi dan mendefinisikan segala kebutuhan yang harus dipenuhi oleh sistem yang akan dikembangkan. Fokus utama dari analisis ini adalah memahami secara mendalam tujuan dan kebutuhan pengguna akhir, baik itu admin, kalab, kaprodi, sekprodi dan aslab. Analisis kebutuhan perangkat lunak dapat dilihat pada Tabel \ref{tab:AnalisisKebutuhanPerangkatLunak}
	      \begin{longtable}{clcc}
		      \caption{Analisis Kebutuhan Perangkat Lunak}
		      \label{tab:AnalisisKebutuhanPerangkatLunak}                                                                                                               \\
		      \hline
		      \multicolumn{1}{l}{\textbf{No}} & \textbf{Perangkat Lunak}     & \multicolumn{1}{l}{\textbf{Versi Minimal}} & \multicolumn{1}{l}{\textbf{Versi Tersedia}} \\ \hline
		      \endfirsthead
		      %
		      \multicolumn{4}{c}%
		      {{\bfseries Table \thetable\ continued from previous page}}                                                                                               \\
		      \hline
		      \multicolumn{1}{l}{\textbf{No}} & \textbf{Perangkat Lunak}     & \multicolumn{1}{l}{\textbf{Versi Minimal}} & \multicolumn{1}{l}{\textbf{Versi Tersedia}} \\ \hline
		      \endhead
		      %
		      \hline
		      \endfoot
		      %
		      \endlastfoot
		      %
		      1                               & Windows                      & W8                                         & W11                                         \\
		      2                               & Balsamiq Mockup              & 4.0.0                                      & 4.7.5                                       \\
		      3                               & Google Chrome                & -                                          & 127.0.6533.100                              \\
		      4                               & MySQL                        & 8.0.0                                      & 8.0.30                                      \\
		      5                               & VS Code                      & 1.71.1                                     & 1.92.1                                      \\
		      6                               & Hypertext Preprocessor (PHP) & 8.0.0                                      & 8.2.16                                      \\
		      7                               & CodeIgniter                  & 4                                          & 4                                           \\ \hline
	      \end{longtable}

	\item Analisis Kebutuhan Perangkat Keras \\
	      Pada tahap analisis ini, peneliti melakukan identifikasi dan mendefinisikan segala kebutuhan yang harus dipenuhi oleh sistem yang dikembangkan. Fokus utama dari analisis ini adalah memahami secara mendalam tujuan dan kebutuhan pengguna akhir, baik admin, kalab, kaprodi, sekprodi, dan aslab. Analisis ini menjadi landasan kritis untuk merancang solusi yang tepat dan memastikan bahwa sistem dapat efektif memenuhi tujuan strategis laboratorium dalam mengelola manajemen tata kelola laboratorium. Analisis kebutuhan ini dapat dilihat pada Tabel \ref{tab:PerangkatKerasPengembang}.
	      \begin{longtable}{clll}
		      \caption{Analisis Kebutuhan Perangkat Keras Pengembang}
		      \label{tab:PerangkatKerasPengembang}                                                                                                                                                                                              \\
		      \hline
		      \textbf{No} & \multicolumn{1}{c}{\textbf{Perangkat Keras}} & \multicolumn{1}{c}{\textbf{Spesifikasi Minimal}}                          & \multicolumn{1}{c}{\textbf{Versi Tersedia}}                                              \\ \hline
		      \endfirsthead
		      %
		      \multicolumn{4}{c}%
		      {{\bfseries Table \thetable\ continued from previous page}}                                                                                                                                                                       \\
		      \hline
		      \textbf{No} & \multicolumn{1}{c}{\textbf{Perangkat Keras}} & \multicolumn{1}{c}{\textbf{Spesifikasi Minimal}}                          & \multicolumn{1}{c}{\textbf{Versi Tersedia}}                                              \\ \hline
		      \endhead
		      %
		      \hline
		      \endfoot
		      %
		      \endlastfoot
		      %
		      1           & Processor                                    & \begin{tabular}[c]{@{}l@{}}Intel Core i3 atau AMD \\ Ryzen 3\end{tabular} & \begin{tabular}[c]{@{}l@{}}AMD Ryzen 5 5600U, 6Cores, \\ 12Threads, 2.3GHz.\end{tabular} \\
		      2           & Memory                                       & 4 GB DDR4                                                                 & 16 GB DDR4-3200 MHz                                                                      \\
		      3           & Storage                                      & \begin{tabular}[c]{@{}l@{}}256 GB SSD atau \\ 500 GB HDD\end{tabular}     & 512 GB M.2 NVMe                                                                          \\
		      4           & Keyboard                                     & \begin{tabular}[c]{@{}l@{}}Standard QWERTY \\ keyboard\end{tabular}       & 6-row, multimedia Fn keys                                                                \\
		      5           & Connection                                   & \begin{tabular}[c]{@{}l@{}}Wi-Fi 802.11n atau \\ Ethernet\end{tabular}    & Wi-Fi® 6                                                                                 \\
		      6           & Monitor                                      & \begin{tabular}[c]{@{}l@{}}14 inch, resolusi \\ 1366x768\end{tabular}     & 13 inc                                                                                   \\ \hline
	      \end{longtable}

	      Dalam spesifikasi perangkat keras yang disarankan pada Sistem \textit{Integrated Laboratory Management Information System} sesuai yang tertera pada Tabel \ref{tab:PerangkatKerasPengembang} sebaiknya memenuhi syarat spesifikasi minimum agar sistem dapat berjalan dengan sempurna.

\end{enumerate}

\section{Perancangan}
Perancangan sistem perlu dilakukan sebelum dilakukan pembuatan sistem. tujuan dari perancangan sistem adalah untuk menentukan, mengorganisir, dan membentuk komponen dari solusi sistem akhir sehingga memiliki \textit{blueprint} untuk membangun sistem.
\subsection{Use Case Diagram}
\textit{\textit{Use case} diagram} terdiri dari \textit{actor}, \textit{use case} dan serta hubungannya. \textit{\textit{Use case} diagram} adalah sesuatu yang penting untuk memvisualisasiakan, menspesifikasikan dan mendokumentasikan kebutuhan perilaku sistem. \textit{\textit{Use case} diagram} digunakan untuk menjelaskan kegiatan apa saja yang dapat dilakukan oleh \textit{user} pengguna sistem yang sedang berjalan \cite{Carstoiu1995}.
\begin{longtable}{clp{8cm}}
	\caption{Deskripsi Aktor}
	\label{tab:DeskripsiAktor}                                                                                                    \\
	\hline
	\textbf{No} & \multicolumn{1}{c}{\textbf{Aktor}} & \multicolumn{1}{c}{\textbf{Deskripsi}}                                     \\ \hline
	\endfirsthead
	%
	\multicolumn{3}{c}%
	{{\bfseries Tabel \thetable\ lanjutan dari halaman sebelumnya}}                                                               \\
	\hline
	\textbf{No} & \multicolumn{1}{c}{\textbf{Aktor}} & \multicolumn{1}{c}{\textbf{Deskripsi}}                                     \\ \hline
	\endhead
	%
	\hline
	\endfoot
	%
	\endlastfoot
	%
	1           & Admin                              & Mengelola penjadwalan laboratorium termasuk lihat, tambah, edit, dan hapus \\
	2           & Kalab                              & Mengelola penjadwalan laboratorium termasuk lihat, tambah, edit, dan hapus \\
	3           & Kaprodi                            & Melihat penjadwalan laboratorium                                           \\
	4           & Sekprodi                           & Melihat penjadwalan laboratorium                                           \\
	5           & Aslab                              & Mengelola penjadwalan laboratorium termasuk lihat, tambah, edit, dan hapus \\ \hline
\end{longtable}
\subsection{\textit{Activity Diagram}}
\subsection{\textit{Class Diagram}}
\subsection{Perancangan \text{Database}}
Perancangan \textit{database} adalah perancangan basis data yang akan digunakan pada sebuah sistem, didasari oleh data perusahaan. Perancangan ini bertujuan agar tiap field data yang memiliki relasi dapat terhubung pada tabel di \textit{database}, sehingga proses pengaksesan data akan dapat telaksana dengan lebih baik. Berikut adalah detail perancangan serta relasi yang ada pada \textit{database} sistem informasi inventaris laboratorium pada Laboratorium Sistem Informasi. Berikut tabel perancangan \textit{database}:

\begin{enumerate}


	\item Perancangan \textit{Database} Tabel Dosen \\
	      \begin{tabular}{lll}
		      Nama \textit{Database} & : & man\_lab  \\
		      Nama Tabel             & : & dosen     \\
		      Field Kunci            & : & id\_dosen \\
	      \end{tabular}

	      Tabel Dosen dirancang untuk menyimpan informasi komprehensif tentang staf pengajar. Struktur tabel ini mencakup berbagai atribut yang diperlukan untuk mengidentifikasi dan mengelola data dosen secara efisien. Berikut adalah penjelasan ilmiah mengenai struktur dan fungsi tabel Dosen:

	      \begin{itemize}
		      \item Tabel ini menggunakan \textit{id\_dosen} sebagai \textit{primary key} dengan tipe data tinyint(4) dan fitur \textit{auto increment}, memastikan setiap dosen memiliki identifikasi unik dalam sistem.
		      \item Atribut 'nama\_dosen' dan 'nip\_dosen' menyimpan informasi identitas dasar dosen, memungkinkan identifikasi personal dalam konteks akademik.
		      \item Field 'jenis\_kelamin' menggunakan tipe data enum untuk memastikan konsistensi data dan memfasilitasi analisis demografis.
		      \item 'email\_dosen' dan 'no\_hp' berfungsi sebagai saluran komunikasi penting, memungkinkan interaksi efektif antara sistem dan dosen.
		      \item Atribut 'nidn' (Nomor Induk Dosen Nasional) menyimpan identifikasi unik dosen yang diakui secara nasional, memfasilitasi integrasi dengan sistem pendidikan tinggi yang lebih luas.
	      \end{itemize}

	      Struktur tabel ini dirancang dengan mempertimbangkan kebutuhan manajemen data dosen yang komprehensif, efisiensi penyimpanan, dan kemudahan dalam pemrosesan dan analisis data.

		      {
			      \fontsize{10}{12}\selectfont
			      \begin{longtable}{l l l l}
				      \caption{Tabel \textit{\textit{Database}} Dosen}
				      \label{admin}                                                                                                  \\
				      \hline
				      \textbf{\textit{Field}} & \textbf{\textit{Type}} & \textbf{\textit{Length}}   & \textbf{\textit{Key}}          \\
				      \hline
				      \endfirsthead

				      \multicolumn{4}{c}{\tablename\ \thetable\ {Tabel \textit{\textit{Database}} Dosen} \space (Tabel lanjutan...)} \\
				      \hline
				      \textbf{\textit{Field}} & \textbf{\textit{Type}} & \textbf{\textit{Length}}   & \textbf{\textit{Key}}          \\
				      \hline
				      \endhead

				      id\_dosen               & tinyint                & 4                          & Primary key (A\_I)             \\
				      nama\_dosen             & varchar                & 100                        &                                \\
				      nip\_dosen              & varchar                & 50                         &                                \\
				      jenis\_kelamin          & enum                   & ('Laki-laki', 'Perempuan') &                                \\
				      email\_dosen            & varchar                & 100                        &                                \\
				      nidn                    & varchar                & 100                        &                                \\
				      no\_hp                  & varchar                & 100                        &                                \\
				      \hline
			      \end{longtable}
		      }

	\item Perancangan \textit{Database} Tabel Matkul \\
	      \begin{tabular}{lll}
		      Nama \textit{Database} & : & man\_lab   \\
		      Nama Tabel             & : & matkul     \\
		      Field Kunci            & : & id\_matkul \\
	      \end{tabular}

	      Tabel Matkul dirancang untuk menyimpan informasi tentang mata kuliah yang ditawarkan dalam program akademik. Struktur tabel ini mencakup berbagai atribut yang diperlukan untuk mengidentifikasi dan mengelola data mata kuliah secara efisien. Berikut adalah penjelasan ilmiah mengenai struktur dan fungsi tabel Matkul:

	      \begin{itemize}
		      \item Tabel ini menggunakan \textit{id\_matkul} sebagai \textit{primary key} dengan tipe data tinyint(4) dan fitur \textit{auto increment}, memastikan setiap mata kuliah memiliki identifikasi unik dalam sistem.
		      \item Atribut 'kode\_matkul' dan 'nama\_matkul' menyimpan informasi identifikasi dasar mata kuliah, memungkinkan pengenalan cepat dalam konteks akademik.
		      \item Field 'sks' dan 'semester' menyimpan informasi penting tentang bobot akademik dan penempatan mata kuliah dalam kurikulum.
		      \item Atribut 'jenis\_matkul' menggunakan tipe data enum untuk mengkategorikan mata kuliah sebagai wajib atau pilihan, memfasilitasi manajemen kurikulum yang fleksibel.
	      \end{itemize}

	      Struktur tabel ini dirancang dengan mempertimbangkan kebutuhan manajemen data mata kuliah yang komprehensif, efisiensi penyimpanan, dan kemudahan dalam pemrosesan dan analisis data kurikulum.

		      {
			      \fontsize{10}{12}\selectfont
			      \begin{longtable}{l l l l}
				      \caption{Tabel \textit{\textit{Database}} Matkul}
				      \label{admin}                                                                                                   \\
				      \hline
				      \textbf{\textit{Field}} & \textbf{\textit{Type}} & \textbf{\textit{Length}} & \textbf{\textit{Key}}             \\
				      \hline
				      \endfirsthead

				      \multicolumn{4}{c}{\tablename\ \thetable\ {Tabel \textit{\textit{Database}} Matkul} \space (Tabel lanjutan...)} \\
				      \hline
				      \textbf{\textit{Field}} & \textbf{\textit{Type}} & \textbf{\textit{Length}} & \textbf{\textit{Key}}             \\
				      \hline
				      \endhead

				      id\_matkul              & tinyint                & 4                        & Primary key (A\_I)                \\
				      kode\_matkul            & varchar                & 50                       &                                   \\
				      nama\_matkul            & varchar                & 100                      &                                   \\
				      sks                     & tinyint                & 4                        &                                   \\
				      semester                & tinyint                & 4                        &                                   \\
				      jenis\_matkul           & enum                   & ('Wajib', 'Pilihan')     &                                   \\
				      \hline
			      \end{longtable}
		      }

	\item Perancangan \textit{Database} Tabel Ruangan \\
	      \begin{tabular}{lll}
		      Nama \textit{Database} & : & man\_lab    \\
		      Nama Tabel             & : & ruangan     \\
		      Field Kunci            & : & id\_ruangan \\
	      \end{tabular}

	      Tabel Ruangan dirancang untuk menyimpan informasi tentang ruangan-ruangan yang tersedia untuk kegiatan akademik. Struktur tabel ini mencakup berbagai atribut yang diperlukan untuk mengidentifikasi dan mengelola data ruangan secara efisien. Berikut adalah penjelasan ilmiah mengenai struktur dan fungsi tabel Ruangan:

	      \begin{itemize}
		      \item Tabel ini menggunakan \textit{id\_ruangan} sebagai \textit{primary key} dengan tipe data tinyint(4) dan fitur \textit{auto increment}, memastikan setiap ruangan memiliki identifikasi unik dalam sistem.
		      \item Atribut 'id\_gedung' berfungsi sebagai \textit{foreign key}, menghubungkan ruangan dengan gedung tempat ruangan tersebut berada, memfasilitasi manajemen lokasi yang terstruktur.
		      \item Field 'nama\_ruangan' menyimpan identifikasi ruangan yang mudah dikenali oleh pengguna.
		      \item Atribut 'deskripsi\_ruangan' memungkinkan penyimpanan informasi tambahan tentang fasilitas atau karakteristik khusus ruangan.
		      \item Field 'gambar\_ruangan' menyimpan path ke file gambar ruangan, memfasilitasi visualisasi dan identifikasi ruangan yang lebih baik.
	      \end{itemize}

	      Struktur tabel ini dirancang dengan mempertimbangkan kebutuhan manajemen data ruangan yang komprehensif, efisiensi penyimpanan, dan kemudahan dalam pemrosesan dan analisis data fasilitas.

		      {
			      \fontsize{10}{12}\selectfont
			      \begin{longtable}{l l l l}
				      \caption{Tabel \textit{\textit{Database}} Ruangan}
				      \label{admin}                                                                                                    \\
				      \hline
				      \textbf{\textit{Field}} & \textbf{\textit{Type}} & \textbf{\textit{Length}} & \textbf{\textit{Key}}              \\
				      \hline
				      \endfirsthead

				      \multicolumn{4}{c}{\tablename\ \thetable\ {Tabel \textit{\textit{Database}} Ruangan} \space (Tabel lanjutan...)} \\
				      \hline
				      \textbf{\textit{Field}} & \textbf{\textit{Type}} & \textbf{\textit{Length}} & \textbf{\textit{Key}}              \\
				      \hline
				      \endhead

				      id\_ruangan             & tinyint                & 4                        & Primary key (A\_I)                 \\
				      id\_gedung              & tinyint                & 4                        & Foreign key                        \\
				      nama\_ruangan           & varchar                & 100                      &                                    \\
				      deskripsi\_ruangan      & text                   &                          &                                    \\
				      gambar\_ruangan         & varchar                & 255                      &                                    \\
				      \hline
			      \end{longtable}
		      }

	\item Perancangan \textit{Database} Tabel Jadwal \\
	      \begin{tabular}{lll}
		      Nama \textit{Database} & : & man\_lab   \\
		      Nama Tabel             & : & jadwal     \\
		      Field Kunci            & : & id\_jadwal \\
	      \end{tabular}

	      Tabel Jadwal dirancang untuk menyimpan informasi tentang penjadwalan kegiatan akademik. Struktur tabel ini mencakup berbagai atribut yang diperlukan untuk mengidentifikasi dan mengelola data jadwal secara efisien. Berikut adalah penjelasan ilmiah mengenai struktur dan fungsi tabel Jadwal:

	      \begin{itemize}
		      \item Tabel ini menggunakan \textit{id\_jadwal} sebagai \textit{primary key} dengan tipe data tinyint(4) dan fitur \textit{auto increment}, memastikan setiap jadwal memiliki identifikasi unik dalam sistem.
		      \item Atribut 'id\_ruangan', 'id\_matkul', dan 'id\_dosen' berfungsi sebagai \textit{foreign key}, menghubungkan jadwal dengan informasi ruangan, mata kuliah, dan dosen terkait, memfasilitasi manajemen jadwal yang terintegrasi.
		      \item Field 'tanggal', 'hari', 'jam\_masuk', dan 'jam\_keluar' menyimpan informasi waktu yang spesifik untuk setiap jadwal.
		      \item Atribut 'deskripsi' memungkinkan penyimpanan informasi tambahan tentang jadwal atau kegiatan.
		      \item Field-field seperti 'kode\_matkul', 'nama\_matkul', 'nama\_dosen', dan 'nama\_ruangan' menyimpan informasi yang redundan namun memfasilitasi akses cepat tanpa perlu melakukan join tabel berulang kali.
	      \end{itemize}

	      Struktur tabel ini dirancang dengan mempertimbangkan kebutuhan manajemen data jadwal yang komprehensif, efisiensi dalam pengambilan data, dan fleksibilitas dalam pengelolaan jadwal akademik.

		      {
			      \fontsize{10}{12}\selectfont
			      \begin{longtable}{l l l l}
				      \caption{Tabel \textit{\textit{Database}} Jadwal}
				      \label{admin}                                                                                                   \\
				      \hline
				      \textbf{\textit{Field}} & \textbf{\textit{Type}} & \textbf{\textit{Length}} & \textbf{\textit{Key}}             \\
				      \hline
				      \endfirsthead

				      \multicolumn{4}{c}{\tablename\ \thetable\ {Tabel \textit{\textit{Database}} Jadwal} \space (Tabel lanjutan...)} \\
				      \hline
				      \textbf{\textit{Field}} & \textbf{\textit{Type}} & \textbf{\textit{Length}} & \textbf{\textit{Key}}             \\
				      \hline
				      \endhead

				      id\_jadwal              & tinyint                & 4                        & Primary key (A\_I)                \\
				      id\_ruangan             & tinyint                & 4                        & Foreign key                       \\
				      id\_matkul              & varchar                & 4                        & Foreign key                       \\
				      id\_dosen               & varchar                & 4                        & Foreign key                       \\
				      tanggal                 & date                   &                          &                                   \\
				      hari                    & varchar                & 50                       &                                   \\
				      jam\_masuk              & time                   &                          &                                   \\
				      jam\_keluar             & time                   &                          &                                   \\
				      deskripsi               & text                   &                          &                                   \\
				      kode\_matkul            & tinyint                & 4                        &                                   \\
				      nama\_matkul            & varchar                & 100                      &                                   \\
				      nama\_dosen             & varchar                & 100                      &                                   \\
				      nama\_ruangan           & varchar                & 100                      &                                   \\
				      \hline
			      \end{longtable}
		      }

	\item Perancangan \textit{Database} Tabel \textit{User} \\
	      \begin{tabular}{lll}
		      Nama \textit{Database} & : & man\_lab          \\
		      Nama Tabel             & : & \textit{user}     \\
		      Field Kunci            & : & id\_\textit{user} \\
	      \end{tabular}

	      Tabel \textit{User} dirancang untuk menyimpan informasi pengguna dalam sistem manajemen laboratorium. Struktur tabel ini mencakup berbagai atribut yang diperlukan untuk mengidentifikasi dan mengautentikasi pengguna, serta mengelola hak akses mereka. Berikut adalah penjelasan ilmiah mengenai struktur dan fungsi tabel \textit{User}:

	      \begin{itemize}
		      \item Tabel ini menggunakan \textit{id\_user} sebagai \textit{primary key} dengan tipe data smallint(4) dan fitur \textit{auto increment}, memastikan setiap pengguna memiliki identifikasi unik dalam sistem.
		      \item Atribut 'nama' dan 'no\_identitas' menyimpan informasi identitas dasar pengguna, memungkinkan identifikasi personal dalam konteks organisasi.
		      \item Field 'foto' menyimpan path ke file gambar profil pengguna, memfasilitasi personalisasi antarmuka pengguna.
		      \item \textit{Username} dan \textit{password\_hash} berfungsi sebagai kredensial autentikasi, dengan \textit{password} dienkripsi untuk keamanan data.
		      \item Atribut 'role\_user' menggunakan tipe data enum untuk mendefinisikan peran pengguna, memungkinkan manajemen hak akses yang efisien dalam sistem.
	      \end{itemize}

	      Struktur tabel ini dirancang dengan mempertimbangkan aspek keamanan, efisiensi penyimpanan data, dan fleksibilitas dalam pengelolaan pengguna sistem.

		      {
			      \fontsize{10}{12}\selectfont
			      \begin{longtable}{l l l l}
				      \caption{Tabel \textit{\textit{Database}} \textit{User}}
				      \label{admin}                                                                                                          \\
				      \hline
				      \textbf{\textit{Field}} & \textbf{\textit{Type}} & \textbf{\textit{Length}} & \textbf{\textit{Key}}                    \\
				      \hline
				      \endfirsthead

				      \multicolumn{4}{c}{\tablename\ \thetable\ {Tabel \textit{\textit{Database}} \textit{User}} \space (Tabel lanjutan...)} \\
				      \hline
				      \textbf{\textit{Field}} & \textbf{\textit{Type}} & \textbf{\textit{Length}} & \textbf{\textit{Key}}                    \\
				      \hline
				      \endhead

				      id\_\textit{user}       & smallint               & 4                        & Primary key (A\_I)                       \\
				      nama                    & varchar                & 50                       &                                          \\
				      foto                    & varchar                & 50                       &                                          \\
				      no\_identitas           & varchar                & 50                       &                                          \\
				      \textit{username}       & varchar                & 100                      &                                          \\
				      \textit{password}\_hash & varchar                & 100                      &                                          \\
				      role\_\textit{user}     & enum                   &                          &                                          \\
				      \hline
			      \end{longtable}
		      }

\end{enumerate}

\subsection{Perancangan Struktur Menu}
\subsection{Perancangan Interface}