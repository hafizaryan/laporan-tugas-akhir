%-----------------------------------------------------------------------------------------------%
%
% Maret 2019
% Template Latex untuk Tugas Akhir Program Studi Sistem informasi ini
% dikembangkan oleh Inggih Permana (inggihjava@gmail.com)
%
% Template ini dikembangkan dari template yang dibuat oleh Andreas Febrian (Fasilkom UI 2003).
%
% Orang yang cerdas adalah orang yang paling banyak mengingat kematian.
%
%-----------------------------------------------------------------------------------------------%


%-----------------------------------------------------------------------------%
\chapter{\babEnam}
%-----------------------------------------------------------------------------%
\thispagestyle{fancy} % Menambahkan nomor halaman di ujung kanan
\section{Kesimpulan}
Berdasarkan hasil penelitian yang dilakukan pada Laboratorium Program Studi Sistem Informasi UIN Suska Riau, maka dapat ditarik kesimpulan yaitu:

\begin{enumerate}
	\item Hasil analisis penelitian ini berhasil mengidentifikasi kekurangan yang ada pada sistem SITARIS.
	\item Hasil pengembangann SITARIS berhasil mengintegrasikan sistem informasi yang berjalan masing-masing menjadi Sistem Manajemen Laboratorium Terintegrasi yang dapat memberikan manfaat yang signifikan bagi pengelolaan laboratorium.
\end{enumerate}

\section{Saran}
Peneliti menyadari dalam pelaksanaan penelitian dan pembuatan laporan maupun sistem masih terdapat celah dan kekurangan. Berdasarkan hal tersebut peneliti membuka diri untuk menerima saran maupun kritik yang membangun bagi peneliti kedepannya. Adapun saran yang ingin peneliti sampaikan diantaranya:

\begin{enumerate}
	\item Meningkatkan fitur dan kapabilitas Sistem Manajemen Laboratorium Terintegrasi agar lebih sesuai dengan kebutuhan pengguna dan perkembangan teknologi.
	\item Menambahkan fitur \textit{backup} dan \textit{recovery system} guna memastikan keamanan data serta mempermudah pemulihan sistem apabila terjadi gangguan atau kehilangan data.
	\item Melakukan penelitian lanjutan tentang efektivitas Metode \textit{Agile Development} dibandingkan dengan metode pengembangan lain dalam konteks sistem manajemen laboratorium.
	\item Membuat dokumentasi proses \textit{sprint} dan \textit{iterative development} sebagai panduan praktis bagi pengembang sistem serupa yang menggunakan pendekatan \textit{Agile}.
\end{enumerate}