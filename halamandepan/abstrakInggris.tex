%-----------------------------------------------------------------------------------------------%
%
% Maret 2019
% Template Latex untuk Tugas Akhir Program Studi Sistem informasi ini
% dikembangkan oleh Inggih Permana (inggihjava@gmail.com)
%
% Template ini dikembangkan dari template yang dibuat oleh Andreas Febrian (Fasilkom UI 2003).
%
% Orang yang cerdas adalah orang yang paling banyak mengingat kematian.
%
%-----------------------------------------------------------------------------------------------%

\chapter*{\MakeUppercase{\textit{\judulInggris}}}

\fontsize{14}{16.8}
\begin{center}
	\vspace{0.3cm}
	\MakeUppercase{\textbf{\penulis}}\\
	\MakeUppercase{\textbf{NIM: \nim}}\\
	\fontsize{12}{14.4}
	\vspace{0.7cm}

	\textit{Date of Final Exam: \tanggalSidangInggris}\\
	\textit{Graduation Period:}\ \ \ \ \ \ \ \ \ \ \ \ \ \ \ \ \ \ \ \ \ \ \ \ \ \ \ \ \ \ \ \ \ \ \ \ \ \

	\vspace{0.7cm}
	\emph{Department of \programStudiInggris}\\
	\textit{\fakultasInggris}\\
	\universitasInggris\\
	\alamatUniversitasInggris\\

	\vspace{0.7cm}
\end{center}

\fontsize{12}{14.4}
\begin{center}\MakeUppercase{\textbf{\emph{Abstract}}}\end{center}

\noindent
\fontsize{10pt}{12pt}\selectfont
\emph{Laboratories play a crucial role in educational institutions by supporting practicum and research activities. Efficient management is essential to ensure the optimal use of resources and facilities. This study aims to develop the Laboratory Inventory Information System (SITARIS) into an Integrated Laboratory Management System in the Information Systems Study Program, UIN Sultan Syarif Kasim Riau. This study emphasizes the use of the Agile Development Method as the main approach in system development. Various obstacles in the previous SITARIS SI system have been resolved and resolved, such as coding errors, dysfunctional borrowing features, and inconsistencies in report formats. The developed system is not only more stable, but also equipped with additional features such as laboratory schedule management. Integration with other supporting systems, such as the Laboratory Visitor Information System (LABVIS) and the Laboratory Assistant Registration Information System (LARIS) which previously ran separately, makes access to the system individual. The development results create an integrated system, and add laboratory scheduling features. This study shows that an integrated laboratory management system can improve operational efficiency, organize inventory management, and improve overall laboratory governance. Thus, this study makes a significant contribution to optimizing laboratory governance in the Information Systems Study Program environment.}\\
\noindent{\emph{\textbf{Keywords:} Agile Development, Integrated Laboratory Management System, Integration, Inventory Management, Laboratory Governance}} \\