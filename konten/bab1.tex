%-----------------------------------------------------------------------------------------------%
%
% Maret 2019
% Template Latex untuk Tugas Akhir Program Studi Sistem informasi ini
% dikembangkan oleh Inggih Permana (inggihjava@gmail.com)
%
% Template ini dikembangkan dari template yang dibuat oleh Andreas Febrian (Fasilkom UI 2003).
%
% Orang yang cerdas adalah orang yang paling banyak mengingat kematian.
%
%-----------------------------------------------------------------------------------------------%

%-----------------------------------------------------------------------------%
\chapter{\babSatu}
%-----------------------------------------------------------------------------%

%-----------------------------------------------------------------------------%
\section{Latar Belakang}
%-----------------------------------------------------------------------------%
Laboratorium merupakan salah satu fasilitas vital dalam institusi pendidikan yang berperan penting dalam mendukung kegiatan praktikum dan penelitian \cite{la2021comparison}. Tata kelola laboratorium yang baik menjadi kunci dalam memastikan penggunaan peralatan dan fasilitas secara optimal dan efisien \cite{jeffrey_s__warren_2017}. Pengelolaan inventaris yang akurat dan efisien adalah bagian integral dari tata kelola ini, yang bertujuan untuk membangun budaya kualitas dalam pendidikan tinggi \cite{abrantes2020governance}. Dengan tata kelola yang efektif, institusi dapat memastikan bahwa semua sumber daya laboratorium digunakan secara maksimal untuk mendukung pengembangan ilmu pengetahuan dan teknologi.

Universitas Islam Negeri Sultan Syarif Kasim Riau sebagai salah satu perguruan tinggi negeri di Indonesia memiliki Program Studi Sistem Informasi di bawah naungan Fakultas Sains dan Teknologi \cite{uin-suska-website}. Program Studi ini dilengkapi dengan fasilitas laboratorium yang menyeluruh untuk mendukung pelaksanaan Tri Dharma Perguruan Tinggi yaitu, pengajaran, penelitian, dan pengabdian kepada masyarakat. Ketiga pilar ini berperan sinergis dalam menjadikan UIN Suska Riau sebagai kontributor signifikan dalam pengembangan ilmu pengetahuan, teknologi, dan kemajuan masyarakat.

Sejak tahun 2002, Program Studi Sistem Informasi telah mengelola tiga laboratorium terpadu di bawah Fakultas Sains dan Teknologi, yaitu Laboratorium Rekayasa Sistem Informasi (RSI), Laboratorium Internet (INT), dan Laboratorium Software Engineering (SE) \cite{lab-si-website}. Laboratorium-laboratorium ini berfungsi tidak hanya sebagai sarana praktikum bagi mahasiswa sesuai kurikulum, tetapi juga sebagai pusat kegiatan riset dan inovasi yang memberikan manfaat substansial bagi civitas akademika, termasuk mahasiswa dan dosen.

Laboratorium-laboratorium Program Studi Sistem Informasi dilengkapi dengan fasilitas yang memadai untuk mendukung pembelajaran mahasiswa dan berbagai kegiatan akademik lainnya (Lampiran B). Untuk memastikan efektivitas dan efisiensi tata kelola laboratorium, dilakukan evaluasi berkala terhadap seluruh aspek fasilitas yang ada \cite{lab-si-website}. Tata kelola laboratorium yang baik sangat penting untuk memantau dan mengelola penggunaan peralatan serta fasilitas laboratorium secara optimal sehingga dapat mendukung kegiatan akademik dan penelitian dengan lebih baik \cite{dongapure2024good}.

Dalam konteks Program Studi Sistem Informasi UIN Suska Riau, tata kelola laboratorium sudah dilakukan dengan beberapa cara yaitu mulai dari pengelolaan inventaris laboratorium sebelumnya dilakukan secara manual, yang mengakibatkan berbagai kendala seperti kesulitan dalam pemantauan dan pengelolaan data inventaris, serta ketidakefisienan dalam pengolahan data. Wild \citeyear{smith2021agile} menunjukkan bahwa pengelolaan inventaris yang tidak efisien dapat menghambat operasional laboratorium. Dalam mengatasi permasalahan tersebut, laboratorium program studi sistem informasi telah mengimplementasikan sistem informasi inventaris bernama SITARIS SI (Lampiran B). Tata kelola laboratorium dalam hal kunjungan juga sudah diterapkan sistem informasi kunjungan bernama \textit{Laboratory Visitor Information System} yang disingkat (LABVIS) pada tahun 2023 (Lampiran B). Hal tersebut bertujuan untuk mempermudah pemantauan dan pengelolaan kunjungan laboratorium secara efisien. Serta sedang dalam proses pengembangan sebuah sistem informasi pendaftaran asisten laboratorium yang bernama \textit{Laboratory Assistant Registration Information System} yang disingkat (LARIS) (Lampiran B).

% Hal ini mendukung tujuan laboratorium Program Studi Sistem Informasi dalam mewujudkan sistem informasi manajemen laboratorium terintegrasi yang bernama \textit{Integrated Laboratory Management Information System} yang disingkat sebagai (ILMIS) yang merupakan salah satu Indikator Kinerja Utama (IKU) Kepala Laboratorium Program Studi Sistem Informasi UIN Suska Riau.

Setelah pengembangan SITARIS SI, manajemen tata kelola laboratorium mengalami peningkatan signifikan dibandingkan dengan proses manual sebelumnya. Sistem yang baru memungkinkan otomatisasi berbagai proses yang sebelumnya memerlukan penanganan manual yang memakan waktu. Pencatatan inventaris yang dahulu dilakukan dengan pembukuan manual, kini dapat dilakukan secara digital dengan sistem pengkodean otomatis yang akurat. Proses pencatatan pendanaan, pengelolaan barang, efisiensi informasi barang, pemeliharaan, dokumentasi, pemusnahan barang dan peminjaman barang serta ruangan yang sebelumnya membutuhkan pencatatan berulang dan validasi manual, kini dapat dikelola melalui sistem dengan \textit{workflow} yang jelas.

Namun demikian, meskipun SITARIS SI telah berperan penting dalam tata kelola laboratorium, perkembangan teknologi yang pesat dan kebutuhan laboratorium yang semakin kompleks telah memunculkan berbagai tantangan baru. Beberapa permasalahan yang teridentifikasi meliputi kesalahan dalam pembuatan kode barang (Lampiran B), disfungsi fitur peminjaman barang dan ruangan (Lampiran B), serta ketidaksesuaian format laporan akhir dengan kebutuhan kepala laboratorium (Lampiran B). Keterbatasan ini berdampak signifikan pada efektivitas manajemen laboratorium secara keseluruhan. Kesalahan dalam pembuatan kode barang menyebabkan kesulitan dalam pelacakan dan inventarisasi aset, sementara disfungsi fitur peminjaman mengakibatkan terhambatnya proses administrasi yang mempengaruhi kelancaran kegiatan praktikum dan penelitian. Selain itu, tidak adanya fitur pengelolaan jadwal laboratorium membuat informasi ketersediaan laboratorium untuk tempat praktikum menjadi tidak \textit{real-time}. Permasalahan-permasalahan ini secara kolektif menjadikan sistem tersebut tidak mampu secara optimal dalam mendukung kebutuhan tata kelola laboratorium.

Dalam pengembangan sistem informasi, berbagai metodologi pengembangan telah diperkenalkan untuk mengatasi tantangan yang timbul, masing-masing dengan kelebihan dan kekurangannya. Salah satu pendekatan yang semakin populer adalah Agile Development, yang telah terbukti efektif dalam menghadapi dinamika dan perubahan kebutuhan pengguna yang cepat. Berbeda dengan metodologi tradisional seperti Waterfall yang bersifat linier dan kurang fleksibel terhadap perubahan, Agile memungkinkan pengembangan sistem secara iteratif dengan penyesuaian terus-menerus berdasarkan umpan balik yang diperoleh dari pengguna. Penelitian-penelitian sebelumnya menunjukkan bahwa pendekatan ini sangat cocok untuk pengembangan sistem informasi yang memerlukan fleksibilitas. Misalnya, Tuunanen dkk. \citeyear{tuunanen2023development} berhasil mengembangkan metode prioritas risiko persyaratan dalam proyek pengembangan sistem informasi dengan pendekatan Agile, sementara Wisnumurti dkk. \citeyear{wisnumurti2022penerapan} membuktikan efektivitas metode ini dalam pengembangan sistem penjualan untuk toko lokal. Selain itu, Galimberti \citeyear{trelles2021agile} dan Zhang dkk. \citeyear{zhang2024establishment} juga menunjukkan keberhasilan implementasi Agile Development dalam lingkungan universitas dan pengembangan arsitektur layanan mikro, yang menyoroti fleksibilitas metode ini dalam berbagai konteks. Meskipun demikian, setiap metodologi pengembangan memiliki kelebihan dan keterbatasannya sendiri. Misalnya, Waterfall lebih cocok untuk proyek dengan persyaratan yang sudah jelas dan tidak berubah, sementara V-Model menekankan pentingnya pengujian pada setiap tahap pengembangan. Namun, kedua metodologi tersebut kurang fleksibel dalam mengakomodasi perubahan yang terjadi di tengah proses. Sementara itu, meskipun DevOps memberikan keunggulan dalam integrasi pengembangan dan operasi secara berkelanjutan, fokusnya lebih pada otomatisasi dan pengelolaan infrastruktur TI, yang berbeda dengan prinsip dasar Agile yang lebih mengutamakan fleksibilitas. Oleh karena itu, Agile Development menjadi pilihan yang tepat dalam pengembangan sistem informasi yang membutuhkan adaptasi cepat, dan kemampuan untuk merespons perubahan secara efektif, yang pada akhirnya dapat meningkatkan kualitas dan produktivitas proyek sistem informasi.

Berdasarkan temuan-temuan tersebut, pengembangan sistem diperlukan untuk meningkatkan kualitas dan fungsionalitas SITARIS SI dalam mendukung tata kelola laboratorium. Pengembangan akan berfokus pada peningkatan efisiensi pengelolaan dengan memperhatikan umpan balik pengguna secara berkelanjutan, mencakup perbaikan fitur yang ada dan penambahan fitur baru yang relevan. Sistem juga akan diintegrasikan dengan teknologi terbaru untuk memastikan kesesuaian dengan perkembangan zaman dan standar yang berlaku. Proses pengembangan akan dilakukan secara bertahap, mulai dari analisis kebutuhan hingga evaluasi dan pemeliharaan, dengan melibatkan partisipasi aktif pengguna untuk memastikan hasil yang sesuai dengan kebutuhan dan harapan.

Oleh karena itu, penelitian ini bertujuan untuk mengembangkan SITARIS SI Program Studi Sistem Informasi UIN Suska Riau menjadi Sistem Manajemen Laboratorium Terintegrasi. Pengembangan akan difokuskan pada peningkatan fungsionalitas dan kinerja sistem secara keseluruhan untuk memenuhi kebutuhan tata kelola laboratorium.
%-----------------------------------------------------------------------------%
\section{Rumusan Masalah}
%-----------------------------------------------------------------------------%
Berdasarkan latar belakang yang telah diuraikan, diperoleh rumusan masalah untuk penelitian ini adalah bagaimana pengembangan SITARIS dan integrasi dengan Sistem Laboratorium lainnya (Website SI, LABVIS, LARIS) menjadi Sistem Manajemen Laboratorium menggunakan Metode Agile Development.

%-----------------------------------------------------------------------------%
\section{Batasan Masalah}
%-----------------------------------------------------------------------------%
Dalam melakukan penelitian diperlukan batasan agar tidak menyimpang dari apa yang direncanakan. Adapun batasan masalah dalam penelitian Tugas Akhir ini adalah sebagai berikut:
\begin{enumerate}
	\item Penelitian ini hanya akan fokus pada pengembangan Sistem Informasi Inventaris Laboratorium (SITARIS SI) menjadi Sistem Manajemen Laboratorium.
	\item Pengembangan sistem akan menggunakan Metode Agile Development.
	\item Penelitian ini tidak mencakup pengembangan perangkat keras laboratorium.
	\item Evaluasi sistem akan dilakukan berdasarkan umpan balik dari pengguna di Program Studi Sistem Informasi UIN Suska Riau.
	\item Penelitian ini hanya akan mencakup laboratorium yang berada di bawah naungan Program Studi Sistem Informasi UIN Suska Riau.
\end{enumerate}

%-----------------------------------------------------------------------------%
\section{Tujuan}
%-----------------------------------------------------------------------------%
Tujuan Tugas Akhir ini adalah:
\begin{enumerate}
	\item Menganalisis SITARIS SI sebagai Sistem Informasi Inventaris Laboratorium untuk mengetahui kekurangan dan kendala yang ada.
	\item Mengembangkan SITARIS SI menjadi Sistem Manajemen Laboratorium Terintegrasi.
\end{enumerate}

%-----------------------------------------------------------------------------%
\section{Manfaat}
%-----------------------------------------------------------------------------%
Hasil penelitian diharapkan dapat menjadi sebuah Sistem Informasi Manajemen Laboratorium yang terintegrasi yang memberikan kemudahan dalam tata kelola laboratorium. Sistem ini diharapkan mampu mengelola tata kelola laboratorium dengan lebih efisien, memantau penggunaan peralatan dan fasilitas secara \textit{real-time}, serta menyediakan laporan yang akurat dan sesuai dengan kebutuhan kepala laboratorium. Selain itu, sistem ini juga diharapkan dapat memfasilitasi peminjaman barang dan ruangan dengan lebih mudah, serta mengelola kunjungan laboratorium secara efektif. Dengan demikian, sistem ini akan mendukung kegiatan praktikum dan penelitian dengan lebih baik, serta meningkatkan kualitas pengelolaan laboratorium secara keseluruhan.
%-----------------------------------------------------------------------------%
\section{Sistematika Penulisan}
%-----------------------------------------------------------------------------%
Sistematika penulisan laporan adalah sebagai berikut:

\textbf{BAB 1. \babSatu}

BAB 1 pada tugas akhir ini berisi tentang: (1) Latar Belakang masalah; (2) Rumusan Masalah; (3) Batasan Masalah; (4) Tujuan; (5) Manfaat; dan (6) Sistematika Penulisan.

\textbf{BAB 2. \babDua}

BAB 2 pada Tugas Akhir ini berisi tentang: (1) Profil Instansi; (2) Pengembangan Sistem Informasi; (3) Manajemen; (4) Manajemen Laboratorium; (5) Laboratorium; (6) SITASI; (7) SmarTA; (8) SIKAPE; (9) SIREPO; (10) Website Lab SI; (11) LABVIS; (12) SITARIS SI; (13) Model Pengembangan Sistem; (14) Unified Modelling Language (UML); (15) Observasi; (16) Wawancara; (17) PHP; (18) Framework; (19) CodeIgniter; (20) Visual Studio Code; (21) Astah; (22) Balsamiq; (23) Database; (24) MariaDB; (25) XAMPP.

\textbf{BAB 3. \babTiga}

BAB 3 pada Tugas Akhir ini berisi tentang: (1) Tahap Perencanaan; (2) Tahap Analisis dan Perancangan; (3) Tahap Implementasi dan Pengujian; (4) Tahap Dokumentasi.

\textbf{BAB 4. \babEmpat}

BAB 4 pada Tugas Akhir ini berisi tentang: (1) Analisis Sistem Berjalan; (2) Analisis Sistem Usulan; (3) Analisis Kebutuhan Sistem; (4) Perancangan.

\textbf{BAB 5. \babLima}

BAB 5 pada Tugas Akhir ini berisi tentang: (1) Kesimpulan; (2) Saran.
