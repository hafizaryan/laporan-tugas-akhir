%-----------------------------------------------------------------------------------------------%
%
% Maret 2019
% Template Latex untuk Tugas Akhir Program Studi Sistem informasi ini
% dikembangkan oleh Inggih Permana (inggihjava@gmail.com)
%
% Template ini dikembangkan dari template yang dibuat oleh Andreas Febrian (Fasilkom UI 2003).
%
% Orang yang cerdas adalah orang yang paling banyak mengingat kematian.
%
%-----------------------------------------------------------------------------------------------%

%-----------------------------------------------------------------------------%
\chapter*{\kataPengantar}
%-----------------------------------------------------------------------------%
\textit{Alhamdulillahi Rabbil’Alamin}, bersyukur kehadirat Allah \textit{Subhanahu Wa Ta’ala} atas segala rahmat dan karunia-Nya sehingga peneliti dapat menyelesaikan Tugas Akhir ini. \textit{Shalawat} beserta salam selalu tercurahkan untuk Nabi Muhammad \textit{Shallallahu ’Alaihi Wa Sallam} dengan mengucapkan \textit{Allahumma Sholli’ala Sayyidina Muhammad Wa’ala Ali Sayyidina Muhammad}. Tugas Akhir ini dibuat sebagai salah satu syarat untuk mendapatkan gelar Sarjana Komputer di Program Studi Sistem Informasi Universitas Islam Negeri Sultan Syarif Kasim Riau. Pada penulisan Tugas Akhir ini, banyak pihak yang telah berperan dalam mendukung dan membantu peneliti. Maka dari itu, ungkapan terima kasih peneliti ucapkan kepada:

\begin{enumerate}
	\item Bapak \rektor\  sebagai Rektor \universitas.
	\item Bapak Dr. Hartono, M.Pd sebagai Dekan \fakultas.
	\item Bapak Eki Saputra, S.Kom., M.Kom sebagai Ketua Program Studi \programStudi.
	\item Ibu Siti Monalisa, ST., M.Kom sebagai Sekretaris Program Studi Sistem Informasi.
	\item Bapak T. Khairil Ahsyar, S.Kom., M.Kom sebagai dosen pembimbing Tugas Akhir ini.
	\item Bapak Arif Marsal, Lc., MA sebagai Ketua Sidang peneliti yang telah banyak memberikan arahan, masukan, nasihat serta motivasi dalam penyelesaian Tugas Akhir ini juga dalam perkuliahan.
	\item Bapak Eki Saputra, S.Kom., M.Kom sebagai Penguji I peneliti yang telah banyak memberikan arahan, masukan, nasihat serta motivasi dalam penyelesaian Tugas Akhir ini juga dalam perkuliahan.
	\item Bapak Muhammad Jazman, S.Kom., M.Infosys sebagai Penguji II peneliti yang telah banyak memberikan arahan, nasihat, masukan serta motivasi dalam penyelesaian Tugas Akhir ini juga dalam perkuliahan.
	\item Ibu Mona Fronita, S.Kom., M.Kom selaku Pembimbing Akademik yang telah memberikan dukungan, arahan, dan masukan kepada peneliti dari awal perkuliahan hingga saat ini.
	\item Segenap Dosen dan Karyawan Program Studi Sistem Informasi Fakultas Sains dan Teknologi Universitas Islam Negeri Sultan Syarif Kasim Riau.
	\item Keluarga kecil peneliti, yaitu Alm. Ahmad Sofyan Siregar selaku ayah peneliti, Ibu Arni Shopiyah Nst selaku ibu peneliti, saudara Hanif Luthfi Siregar dan saudari Nia Fanesa Siregar selaku adik kandung peneliti, yang selalu memberikan semangat baik berupa moril maupun materil, motivasi dan yang selalu mendoakan peneliti.
	\item Saudari Rahma Yulia Fani yang telah banyak memberikan motivasi, dukungan dan semangat kepada peneliti dalam menyelesaikan Tugas Akhir ini.
	\item Seluruh Bapak, Ibu dan juga teman-teman di Information System Networking Club Research (ISNC Research). Terima kasih sudah memberikan dukungan untuk menyelasaikan Tugas Akhir ini.
	\item Teman-teman berproses di Bangkit Academy yaitu Dani Harmade, Rahmat Afriyanto, Fajri Nurhadi, Hapid Ramdani, Erliandika Syahputra, dan Ahmeid Aqeil yang telah menjadi bagian dari cerita perkuliahan ini.
	\item Semua teman-teman Sistem Informasi angkatan 2021 yang selalu mendukung, berbagi informasi, dan membantu peneliti dalam menjalankan masa perkuliahan menjadi lebih mudah dan semua pihak yang namanya tidak dapat disebutkan yang telah banyak membantu dalam pelaksanaan serta penyelesaian Tugas Akhir ini.
\end{enumerate}

Semoga segala doa dan dorongan yang telah diberikan selama ini menjadi amal kebajikan dan mendapat balasan setimpal dari Allah \textit{Subhanahu Wa Ta’ala}. Peneliti menyadari bahwa penulisan Tugas Akhir ini masih terdapat kekurangan dan jauh dari kata sempurna. Peneliti berharap untuk kritik dan saran yang membangun yang dapat disampaikan melalui email \href{mailto:12150310904@students.uin-suska.ac.id}{12150310904@students.uin-suska.ac.id} atau \href{mailto:hafizaryansiregar@gmail.com}{hafizaryansiregar@gmail.com} untuk Tugas Akhir ini dan semoga Laporan Tugas Akhir ini bermanfaat bagi kita semua. Akhir kata peneliti ucapkan terima kasih.

\textit{Wassalamu’alaikum Warahmatullahi Wabarakaatuh.}

\vspace*{0.1cm}

% \begin{flushright}
% 	\kota, \tanggalPersetujuan\\
% 	Peneliti,\\
% 	\vspace{2cm}
% 	\textbf{\underline{\penulis}\\
% 		NIM. \nim}

% \end{flushright}

\begin{flushright}
	\kota, \tanggalPersetujuan\\
	Peneliti,\\
	\vspace{2cm}
	\textbf{\underline{\penulis}\\
		\vspace{-0.15cm}
		NIM. \nim}

\end{flushright}
