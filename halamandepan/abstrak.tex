%-----------------------------------------------------------------------------------------------%
%
% Maret 2019
% Template Latex untuk Tugas Akhir Program Studi Sistem informasi ini
% dikembangkan oleh Inggih Permana (inggihjava@gmail.com)
%
% Template ini dikembangkan dari template yang dibuat oleh Andreas Febrian (Fasilkom UI 2003).
%
% Orang yang cerdas adalah orang yang paling banyak mengingat kematian.
%
%-----------------------------------------------------------------------------------------------%

\chapter*{\MakeUppercase{\judul}}

\fontsize{14}{16.8}
\begin{center}
	\vspace{0.3cm}
	\MakeUppercase{\textbf{\penulis}}\\
	\MakeUppercase{\textbf{NIM: \nim}}\\
	\fontsize{12}{14.4}
	\vspace{0.7cm}

	Tanggal Sidang: \tanggalSidang\\
	Periode Wisuda:\ \ \ \ \ \ \ \ \ \ \ \ \ \ \ \ \ \ \ \ \ \ \ \ \ \ \ \ \ \ \ \ \ \ \ \ \ \

	\vspace{0.7cm}
	Program Studi \programStudi\\
	\fakultas\\
	\universitas\\
	\alamatUniversitas\\

	\vspace{0.7cm}
\end{center}

\fontsize{12}{14.4}
\begin{center}\MakeUppercase{\textbf{Abstrak}}\end{center}

\noindent
\fontsize{10pt}{12pt}\selectfont
Laboratorium memainkan peran krusial dalam institusi pendidikan dengan mendukung kegiatan praktikum dan penelitian. Pengelolaan yang efisien sangat penting untuk memastikan optimalisasi penggunaan sumber daya dan fasilitas. Penelitian ini bertujuan mengembangkan Sistem Informasi Inventaris Laboratorium (SITARIS) menjadi Integrated Laboratory Management Information System (ILMIS) pada Program Studi Sistem Informasi, UIN Sultan Syarif Kasim Riau. Penelitian ini menekankan penggunaan Metode Agile Development sebagai pendekatan utama dalam pengembangan sistem. Berbagai keterbatasan pada sistem SITARIS SI sebelumnya berhasil diidentifikasi dan diselesaikan, seperti kesalahan pengkodean, disfungsi fitur peminjaman, dan ketidaksesuaian format laporan. Sistem hasil pengembangan tidak hanya lebih stabil, tetapi juga dilengkapi dengan fitur tambahan seperti manajemen jadwal laboratorium. Penggabungan dengan sistem pendukung lainnya, seperti Laboratory Visitor Information System (LABVIS) dan Laboratory Assistant Registration Information System (LARIS) yang sebelumnya berjalan secara mandiri, kini memungkinkan akses sistem dilakukan secara terintegrasi. Hasil pengembangan menjadikan sebuah sistem yg terintegrasi, dan menambahkan fitur penjadwalan laboratoium. Penelitian ini menunjukkan bahwa ILMIS dapat meningkatkan efisiensi operasional, menyederhanakan pengelolaan inventaris, serta memperbaiki tata kelola laboratorium secara keseluruhan. Dengan demikian, penelitian ini memberikan kontribusi yang signifikan dalam mengoptimalkan tata kelola laboratorium di lingkungan Program Studi Sistem Informasi.\\
\noindent{\textbf{Kata Kunci:} Agile Development, ILMIS, Integrasi, Manajemen Inventaris, Tata Kelola laboratorium} \\