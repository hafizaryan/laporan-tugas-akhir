%-----------------------------------------------------------------------------------------------%
%
% Maret 2019
% Template Latex untuk Tugas Akhir Program Studi Sistem informasi ini
% dikembangkan oleh Inggih Permana (inggihjava@gmail.com)
%
% Template ini dikembangkan dari template yang dibuat oleh Andreas Febrian (Fasilkom UI 2003).
%
% Orang yang cerdas adalah orang yang paling banyak mengingat kematian.
%
%-----------------------------------------------------------------------------------------------%


%-----------------------------------------------------------------------------%
\chapter{\babEnam}
%-----------------------------------------------------------------------------%
\section{Kesimpulan}
Berdasarkan hasil penelitian yang dilakukan pada Laboratorium Program Studi Sistem Informasi UIN Suska Riau, maka dapat ditarik kesimpulan yaitu:

\begin{enumerate}
	\item Penelitian ini berhasil mengidentifikasi kekurangan yang ada pada sistem SITARIS SI, Selain itu, penelitian ini juga berhasil mengembangkan SITARIS SI menjadi Sistem Manajemen Laboratorium Terintegrasi.
	\item Sistem Manajemen Laboratorium Terintegrasi yang dikembangkan dalam penelitian ini diharapkan dapat memberikan manfaat yang signifikan bagi pengelolaan laboratorium.
\end{enumerate}

\section{Saran}
Peneliti menyadari dalam pelaksanaan penelitian dan pembuatan laporan maupun sistem masih terdapat celah dan kekurangan. Berdasarkan hal tersebut peneliti membuka diri untuk menerima saran maupun kritik yang membangun bagi peneliti kedepannya. Adapun saran yang ingin peneliti sampaikan diantaranya:

\begin{enumerate}
	\item Meningkatkan fitur dan kapabilitas Sistem Manajemen Laboratorium Terintegrasi agar lebih sesuai dengan kebutuhan pengguna dan perkembangan teknologi.
	\item Melakukan monitoring dan pemeliharaan rutin terhadap sistem untuk memastikan efisiensi operasional dan efektivitas pengelolaan manajemen laboratorium tetap terjaga.
	\item Melakukan pelatihan kepada pengguna sistem agar dapat memanfaatkan semua fitur yang ada pada Sistem Manajemen Laboratorium Terintegrasi dengan baik.
	\item Melakukan pengujian sistem secara berkala untuk memastikan bahwa sistem tetap berfungsi dengan baik dan sesuai dengan kebutuhan pengguna.
	\item Menambahkan fitur backup dan recovery system guna memastikan keamanan data serta mempermudah pemulihan sistem apabila terjadi gangguan atau kehilangan data.
\end{enumerate}