%-----------------------------------------------------------------------------------------------%
%
% Maret 2019
% Template Latex untuk Tugas Akhir Program Studi Sistem informasi ini
% dikembangkan oleh Inggih Permana (inggihjava@gmail.com)
%
% Template ini dikembangkan dari template yang dibuat oleh Andreas Febrian (Fasilkom UI 2003).
%
% Orang yang cerdas adalah orang yang paling banyak mengingat kematian.
%
%-----------------------------------------------------------------------------------------------%

%-----------------------------------------------------------------------------%
\chapter{\babSatu}
%-----------------------------------------------------------------------------%

%-----------------------------------------------------------------------------%
\section{Latar Belakang}
%-----------------------------------------------------------------------------%
Program Studi Sistem Informasi merupakan salah satu program studi yang berada di Fakultas Sains dan Teknologi UIN Suska Riau. Program Studi Sistem Informasi dilengkapi dengan laboratorium yang mendukung pelaksanaan Tri Dharma Perguruan Tinggi, yang dalam konteks pendidikan tinggi di Indonesia, meliputi pengajaran, penelitian, dan pengabdian kepada masyarakat. Pilar-pilar ini secara kolektif menjadikan perguruan tinggi sebagai kontributor dalam pengembangan pengetahuan, teknologi, dan masyarakat secara keseluruhan. Hal ini juga mencakup sesi praktikum yang bermanfaat bagi mahasiswa dan dosen.

Laboratorium merupakan tempat yang digunakan mahasiswa untuk melakukan kegiatan pengujian, riset ilmiah, praktikum, serta penelitian (Putri, 2013). Program Studi Sistem Informasi memiliki fasilitas infrastruktur pendukung Tridharma Perguruan Tinggi yang baik, salah satunya adalah laboratorium terpadu di bawah Fakultas Sains dan Teknologi yang dikelola oleh Program Studi Sistem Informasi sejak tahun 2002. Terdapat tiga laboratorium yang dikelola oleh Program Studi Sistem Informasi, yaitu Laboratorium Rekayasa Sistem Informasi (RSI), Laboratorium Internet (INT), dan Laboratorium Software Engineering (SE) (Ahsyar, 2023). Ketiga laboratorium tersebut merupakan sumber daya berharga yang dapat dimanfaatkan secara efektif untuk mencapai tujuan universitas dan menghasilkan lulusan yang kompeten di Program Studi Sistem Informasi melalui pendidikan, penelitian, dan pengabdian kepada masyarakat, dengan tetap mengintegrasikan nilai-nilai keislaman. Laboratorium-laboratorium ini tidak hanya digunakan untuk praktikum mahasiswa sesuai kurikulum, tetapi juga mendukung berbagai kegiatan mahasiswa dan dosen untuk meningkatkan pengetahuan di bidang Sistem Informasi. Laboratorium-laboratorium Program Studi Sistem Informasi dilengkapi dengan fasilitas yang memadai untuk mendukung pembelajaran mahasiswa. Evaluasi terhadap fasilitas yang ada di laboratorium-laboratorium Program Studi Sistem Informasi dilakukan untuk meningkatkan pengalaman belajar mahasiswa dalam memahami materi, termasuk manajemen inventaris.

Manajemen inventaris merupakan salah satu bentuk pengawasan barang - barang yang ada di Laboratorium Program Studi Sistem Informasi di UIN Suska Riau. Tujuan dari manajemen ini untuk memantau jumlah, kondisi, dan status barang yang ada di laboratorium. Pada penelitian sebelumnya, proses pengelolaan inventaris sudah dilakukan secara terkomputerisasi berkat dikembangkannya sistem bernama SITARIS SI yang dikembangkan pada penelitian kerja praktek mini proyek tahun 2023, hal tersebut memberikan kemudahan bagi pengelola laboratorium termasuk Kepala Laboratorium dan Asisten Laboratorium dalam pengelolaan barang inventaris, namun dalam penerapannya yang kurang lebih 9 bulan masih terdapat kekurangan – kekurangan dalam sistem informasi tersebut. Oleh karena itu peneliti akan melakukan penelitian Pengembangan SITARIS SI dengan menggunakan Medote McCALL untuk menguji kualitas SITARIS pada aspek

%-----------------------------------------------------------------------------%
\section{Perumusan Masalah}
%-----------------------------------------------------------------------------%
Rumusan masalah penelitian ini adalah bagaimana mengevaluasi kualitas SITARIS SI menggunakan metode Mccal Quality Model untuk dapat menenentukan tingkat kualitas sistem dan kemudahan akses bagi pengguna serta memberikan rekomendasi untuk pengembangan sistem selanjutnya.

%-----------------------------------------------------------------------------%
\section{Batasan Masalah}
%-----------------------------------------------------------------------------%
Dalam melakukan penelitian diperlukan batasan agar tidak menyimpang dari apa yang direncanakan. Adapun batasan masalah dalam penelitian Tugas Akhir, yaitu:
\begin{enumerate}
	\item Penelitian ini menggunakan Model McCall sebagai untuk mengukur tingkat kualitas dari perspektif Product Operation dan Product Revision.
	\item Evaluasi pada SITARIS SI menggunakan 5 dari 11 faktor, yaitu: Correctness, Reliability, Efficiency, Integrity dan Usability dan Maintainability.
	\item Teknik pengujian kualitas menggunakan Kuesioner, Wawancara, Black Box Testing, dan Manual Testing.
	\item Teknik sampling yang digunakan adalah Probability Sampling yaitu dengan \textit{Proportionate Stratified Random sampling}.
\end{enumerate}

%-----------------------------------------------------------------------------%
\section{Tujuan}
%-----------------------------------------------------------------------------%
Tujuan tugas akhir ini adalah:
\begin{enumerate}
	\item Untuk mengetahui tingkat kualitas website SITARIS SI menggunakan Model McCall pada faktor Product Operation (Correctness, Reliability, Efficiency, Integrity dan Usability) dan pada faktor Product Revision (Maintainability).
	\item Membuat tabel rekomendasi solusi berdasarkan hasil evaluasi kualitas website SITARIS SI guna meningkatkan kualitas website pengelolaan inventaris di Laboratorium Sistem Informasi.	
\end{enumerate}

%-----------------------------------------------------------------------------%
\section{Manfaat}
%-----------------------------------------------------------------------------%
Hasil penelitian diharapkan dapat memberikan manfaat sebagai berikut
\begin{enumerate}
	\item Dapat mengetahui kualitas website SITARIS SI sebagai sistem informasi inventaris berdasarkan metode Mccall Quality Model.
	\item Sebagai bahan pertimbangan bagi pihak Laboratorium Sistem Informasi guna melakukan perbaikan terhadap kualitas website apabila selama diterapkan website ini masih terdapat kekurangan.
\end{enumerate}

%-----------------------------------------------------------------------------%
\section{Sistematika Penulisan}
%-----------------------------------------------------------------------------%
Sistematika penulisan laporan adalah sebagai berikut:

\textbf{BAB 1. \babSatu}

BAB 1 pada tugas akhir ini berisi tentang: (1) latar belakang masalah; (2) rumusan masalah; (3) batasan masalah; (4) tujuan; (5) manfaat; dan (6) sistematika penulisan.

\textbf{BAB 2. \babDua}

BAB 2 pada Tugas Akhir ini berisi tentang: (1) Penelitian Terdahulu; (2) Laboratorium Program Studi Sistem Informasi; (3) Visi dan Misi Laboratorium Program Studi Sistem Informasi; (4) Struktur Organisasi Laboratorium Program Studi Sistem Informasi; (5) SITARIS SI; (6)Evaluasi; (7) Sistem Informasi; (8) Kualitas Sistem; (9) Faktor Kualitas Mccall; (10) Skala likert; (11) Populasi dan Sampel; (12) Teknik Sampling; (13) SPSS; (14) Rumus Pengukuran Mccall.


\textbf{BAB 3. \babTiga}

BAB 3 pada Tugas Akhir ini berisi tentang: (1) Tahap Perencanaan; (2)
Tahap Pengumpulan Data; (3) Tahap Pengolahan Data; (4) Tahap Analisis dan
Hasil; (5) Hasil.


\textbf{BAB 4. \babEmpat}

BAB 4 pada Tugas Akhir ini berisi tentang: (1) Analisis Kondisi SITARIS SI Saat ini; (2) Analisis Pengolahan Data Kuesioner; (3) Analisis Uji Validitas dan Reliabilitas; (4) Analisis Pengukuran Mccall Quality Model; (5) Analisis Kualitas SITARIS SI; (6) Analisis Hasil Penelitian; (7) Rekomendasi.

\textbf{BAB 5. \babLima}

BAB 5 pada Tugas Akhir ini berisi tentang: (1) Kesimpulan; (2) Saran.

% \textbf{BAB 6. \babEnam}

% BAB 6 pada tugas akhir ini berisi tentang: DST.