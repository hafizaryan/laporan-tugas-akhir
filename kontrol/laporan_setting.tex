%-----------------------------------------------------------------------------------------------%
%
% Maret 2019
% Template Latex untuk Tugas Akhir Program Studi Sistem informasi ini
% dikembangkan oleh Inggih Permana (inggihjava@gmail.com)
%
% Template ini dikembangkan dari template yang dibuat oleh Andreas Febrian (Fasilkom UI 2003).
%
% Orang yang cerdas adalah orang yang paling banyak mengingat kematian.
%
%-----------------------------------------------------------------------------------------------%

%-----------------------------------------------------------------------------------------------%
% Dilarang mengedit file ini, karena dapat merubah format penulisan
%-----------------------------------------------------------------------------------------------%

\var{\fakultas}{Fakultas Sains dan Teknologi}
\var{\fakultasInggris}{Faculty of Science and Technology}

\var{\programStudi}{Sistem Informasi}
\var{\programStudiInggris}{Information System}

\var{\gelar}{Sarjana Komputer}

\var{\universitas}{Universitas Islam Negeri Sultan Syarif Kasim Riau}
\var{\universitasInggris}{\emph{State Islamic University of Sultan Syarif Kasim Riau}}

\var{\kota}{Pekanbaru}

\var{\alamatUniversitas}{Jl. Soebrantas, No. 155, Pekanbaru}
\var{\alamatUniversitasInggris}{\emph{Soebrantas Street, No. 155, Pekanbaru}}

\var{\kaprodi}{Eki Saputra, S.Kom., M.Kom.}
\var{\kaprodinip}{198307162011011008}

\var{\dekan}{Dr. Hartono, M.Pd.}
\var{\dekannip}{196403011992031003}

\var{\rektor}{Prof. Dr. Hairunas, M. Ag}
\var{\rektorSatu}{Prof. Dr. Hj. Helmiati, M.Ag}
\var{\rektorDua}{Prof. Dr. H. Mas’ud Zein, M.Pd}
\var{\rektorTiga}{Prof. Edi Erwan, S.Pt., M.Sc., Ph.D}

%-----------------------------------------------------------------------------%
% Judul BAB
%-----------------------------------------------------------------------------%
%

\Var{\lembarPersetujuan}{LEMBAR PERSETUJUAN}
\Var{\lembarPengesahan}{LEMBAR PENGESAHAN}
\Var{\kataPengantar}{Kata Pengantar}
\Var{\babSatu}{Pendahuluan}
\Var{\babDua}{Landasan Teori}
\Var{\babTiga}{Metodologi Penelitian}
\Var{\babEmpat}{Analisis dan Perancangan}
\Var{\babLima}{Implementasi dan Pengujian}
\Var{\babEnam}{Penutup}